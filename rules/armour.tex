Armour reduces Damage, but only covers some parts of the body.
The more \gls{dr} the armour has, the more Damage it takes away, and the more armour the character wears, the more dice-results will be covered by the armour. 

\begin{itemize}
  \item
  Armour removes an amount of Damage equal to its \gls{dr}.
  \item
  Most armour has a \gls{covering} of `3', meaning that it covers most of the body (but not all) and only covers 3 steps of the dice.
  \begin{itemize}
    \item
    When a player rolls 3 steps over an enemy's \gls{tn}, they have struck an unarmoured area, making \pgls{vitalShot}.
    This ignores the opponent's \gls{dr}.
    \item
    When a player rolls 3 steps under an enemy's \gls{tn}, they receive \pgls{vitalShot}, and their character's \gls{dr} does not apply.
  \end{itemize}
  \item
  Full armour can provide better \gls{covering}.
\end{itemize}
