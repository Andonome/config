Armour reduces Damage, but only covers some parts of the body.
The more \gls{dr} the armour has, the more Damage it takes away, and the more armour the character wears, the more dice-results will be covered by the armour. 

\begin{itemize}
  \item
  \Glspl{pc} armour usually covers two steps of dice, plus the draw.
  So when \glspl{pc} miss by only 1 or 2 steps, they can reduce Damage by the armour's \gls{dr}.
  \item
  If they miss an attack roll by 3 or more, the armour does nothing, and they receive a \textit{Vitals Shot}.
  \item
  Similarly, \glspl{pc} rolling over an opponent's Attack score, and \emph{then} above their armour's \textit{Covering}, ignore all \gls{dr} as their weapon finds bare flesh.
  \item
  Full armour can provide more covering.
  \item
  \Glspl{fp} apply only after \gls{dr}.
\end{itemize}
