\providecommand\glossarytitle{Lexicon}

%%%%% Alter topicmcols style

% Header line above topic
\renewcommand*{\glstopicPreSkip}{\bigLine\medskip\needspace{10\baselineskip}}
\renewcommand*{\glstopicInit}{\setlength{\parskip}{10pt}}

\newglossary*{people}{People}
\glssetcategoryattribute{people}{indexonlyfirst}{true}

\newglossary*{mech}{Mechanics}
\glssetcategoryattribute{mech}{indexonlyfirst}{true}

\renewcommand*{\acronymtype}{mech}

\glssetcategoryattribute{indexed}{dualindex}{true}
\glsdefpostdesc{indexed}{.\space\glsentrytext{squash}}

\glsdefpostdesc{general}{.}
\glssetcategoryattribute{general}{glossnamefont}{textbf}
\glsdefpostdesc{people}{.\space\adforn{15}\space}
\glsdefpostdesc{rules}{.\space\adforn{35}\space}
\glsdefpostdesc{god}{.}
\glsdefpostdesc{acronym}{.}
\glsdefpostdesc{symbol}{.}

\glssetcategoryattribute{profession}{glossnamefont}{textit}
\glsdefpostdesc{profession}{.\\\textcolor{\pageSideColor}{\hrulefill}}
\glssetcategoryattribute{profession}{noindex}{true}
\glssetcategoryattribute{profession}{nonumberlist}{true}

\glssetcategoryattribute{rank}{glossnamefont}{textbf}
\glsdefpostdesc{rank}{.}

\glssetcategoryattribute{exposition}{glossdescfont}{textit}
\glssetcategoryattribute{exposition}{nonumberlist}{false}
\glsdefpostdesc{exposition}{.\space\adforn{3}\space}

\glssetcategoryattribute{god}{glossnamefont}{textsf}

\newignoredglossary*{peeps}

\glssetcategoryattribute{rules}{nonumberlist}{true}
\glssetcategoryattribute{rules}{noindex}{true}
\glssetcategoryattribute{rules}{glossnamefont}{textsc}
\glssetcategoryattribute{abbreviation}{glossnamefont}{textit}

\glsdefpostdesc{location}{.\\\textcolor{\pageSideColor}{\flourish\hrulefill}}
\glssetcategoryattribute{location}{glossnamefont}{textsc}

\makeglossaries

% We only want to show full descriptions (e.g. 'PC (Player Character)') in some books. These books make the file '.switch-gls', but most use nothing.

\IfFileExists{.switch-gls}{
  \setabbreviationstyle[acronym]{long-short-sc-desc}
}{
  \setabbreviationstyle[acronym]{short-sc-desc}
}


%%%%%%%%%% Preambles %%%%%%%%%%

\setglossarypreamble{From the safety of a town, this medieval world looks familiar, but people rarely go to war, and nobody has heard of a plague.
Nobody goes hungry outside of a town, as all forests bloom with roots, fruits, and monsters.

And the monsters wander through a generous forest.}

\setglossarypreamble[mech]{\label{glosPreamble}BIND runs on a minimalist set of open-ended, extensible rules.
This collection is all you will need for this book.}

%%%%%%%%%% Meta Terminology %%%%%%%%%%

\newacronym[
  description={a grunge-fantasy RPG, which has no house-rules, and never will},
  sort={T0},
  name={BIND},
  nonumberlist,
  type={peeps},
]{bind}{BIND}{BIND Is Not DnD}

\newacronym[
  description={books work a little like `print on demand', but it's faster, and cheaper, and if you don't like the paper-quality, you have only yourself to blame},
  prefix={a\space},
  parent={bind},
  ]{piy}{PiY}{Print it Yourself}

\newacronym[
  description={means that the file is for printing, not for reading on a screen},
  prefix={a\space},
  parent={bind},
  nonumberlist,
  ]{pdf}{pdf}{Printable Document Format}

% Traits

\longnewglossaryentry{trait}{
  sort={T1.1},
  name={Traits},
  text={Trait},
  prefix={a\space},
  type={mech},
  description={},
  }

\longnewglossaryentry{attribute}{
  name={Attributes},
  text={Attribute},
  prefix={an\space},
  type={mech},
  parent={trait},
  first={\textit{Attribute}},
  description={describe the body and mind.
    \par
    \vspace{1em}
    \noindent
    \begin{minipage}{\linewidth}
    \begin{description}
      \item[Strength:]
      muscle, brawn, toughness, height
      \item[Dexterity:]
      finesse, coördination, balance
      \item[Speed:]
      velocity, tendons, vim
      \item[Intelligence:]
      memory, logic, tenacity, cunning
      \item[Wits:]
      alacrity, levity, attention, acumen
      \item[Charisma:]
      gravitas, glamour, confidence, symmetry
    \end{description}
    \end{minipage}
    Players can remove penalties with minimal \glsentrytext{xp} expenditure, but the price hike after that grows steeply},
}

\longnewglossaryentry{skill}{
  name={Skills},
  text={Skill},
  prefix={a\space},
  type={mech},
  parent={trait},
  nonumberlist,
  description={each help with many different tasks, depending on the \glsentrytext{attribute} paired with.
  \roll{Intelligence}{Larceny} lets the character open a door, while \roll{Dexterity}{Larceny} lets them pick a pocket},
  }


% Time

\longnewglossaryentry{campaign}{
  sort={T3.1},
  name={Chronicles},
  text={Chronicle},
  prefix={a\space},
  type={mech},
  description={The Chronicle is the game and the players, it tells the story of the troupe, but not of any particular \glsentrytext{pc}.
  Each week which passes in our world, about four weeks pass in \glsentrytext{fenestra}.
  During the game, the \glsentrytext{gm} and players set the pace of any scene, but the session never covers more than thirty days},
  }

\newacronym[
  description={-- one of the characters run by the people playing the game},
  parent={campaign},
  sort={A1},
  prefix={a\space},
  ]{pc}{PC}{Player Character}

\longnewglossaryentry{gm}{
  prefix={a\space},
  name={The Judge},
  text={Judge},
  sort={A2},
  type={mech},
  parent={campaign},
  nonumberlist,
  description={rolls encounters, interprets the rules, and forgets to bring enough pencils},
}

\newacronym[
  description={ -- anyone in the world played by the \glsentrytext{gm} rather than a player},
  prefix={an\space},
  sort={A3},
  prefixfirst={a\space},
  parent={campaign},
  ]{npc}{NPC}{Non-Player Character}

\longnewglossaryentry{downtime}{
  name={Downtime},
  text={Downtime},
  prefix={a\space},
  parent={campaign},
  type={mech},
  description={covers the time between scenes and sessions, letting characters train, heal, and drink.
  Characters heal a number of \glsfmtplural{hp} each week equal to half their current total (minimum 1), along with all \glsfmtplural{mp} and~\glsfmtplural{fp}},
  }

\longnewglossaryentry{sq}{
  name={Side Quests},
  text={Side~Quest},
  prefix={a\space},
  parent={campaign},
  type={mech},
  description={are \glsentrytext{bind}'s way of weaving emergent stories.
  Each one introduces itself to the \glsentrytext{gm} with a summary of its scenes, each segmented into \glsfmtplural{region}},
  }

\longnewglossaryentry{segment}{
  name={Segments},
  text={Segment},
  prefix={a\space},
  parent={sq},
  type={mech},
  description={describe events which can happen at any time, anywhere within their \glsentrytext{region}.
  Some \glsfmtplural{sq} have all of their Segments in a single \glsentrytext{region}, while others have Segments which jump between \glsfmtplural{region}.

  Some Segments have a `\glsentrytext{squash}' symbol, meaning they should be run at the same time as the next available Segment in the \glsentrytext{region}.

  Once the Segments concludes, it should be marked as done with a `\glsentrytext{sqr}\vphantom{\hspace{-.5em}\textbackslash}', and the next Segment in the same \glsentrytext{sq} becomes ready (`\glsentrytext{sqr}')},
  }

\longnewglossaryentry{interval}{
  name={Intervals},
  text={Interval},
  prefix={an\space},
  parent={campaign},
  sort={Interval},
  type={mech},
  nonumberlist,
  description={divide the day into four parts -- morning (\showInterval{0}), afternoon (\showInterval{1}), evening (\showInterval{2}), and night (\showInterval{3}).
  After each Interval, each \glsentrytext{pc} regenerates:
    \par
    \noindent
    \begin{itemize}
      \item
      Resting characters remove 1~\glsentrytext{ep}.
      \item
      The \glsentrytext{gm} rolls $1D6$ -- everyone gains that many \glspl{fp}.
      \item
      The wind brings \glspl{mp}, and each point goes towards whoever has the most empty \glspl{mp}.
    \end{itemize}
    \manaRegenChart
    Each day, everyone must eat and sleep, or take two~\glsfmtplural{ep}},
}

% Stories

\longnewglossaryentry{storypoint}{
  name={Story Points},
  text={Story Point},
  type={mech},
  prefix={a\space},
  parent={campaign},
  description={allow players to declare that some part of their backstory arrives on scene to help the situation.
  This could be knowing an ally, an obscure fact, or another language.

  Spending a Story Point grants 5~\glsfmtlongpl{xp}},
}

\newacronym[
  description={come from each character's Code.
  Spend XP to raise any \glsentrytext{trait}.
    \noindent
    \begin{boxtable}[XYccc]
         \textbf{Trait} & \textit{Remove Penalty}  & \textbf{First}  & \textbf{Second} & \textbf{Third}\\
         \hline
         \glsentrytext{skill}      & --- &  5 & 10 & 15 \\
         Knack                     & --- &  5 & 10 & 15 \\
         Combat \glsfmttext{skill} & --- & 10 & 20 & 30 \\
         \glsentrytext{attribute}  &   5 & 10 & 20 & 40 \\
    \end{boxtable}
  The racial limits adjust these numbers as usual},
  shortplural={XP},
  prefix={an\space},
  name={Experience Points (XP)},
  parent={storypoint},
  nonumberlist,
  ]{xp}{XP}{Experience Point}

\longnewglossaryentry{characterPool}{
  name={Character Pool},
  prefix={a\space},
  type={mech},
  category={rules},
  parent={storypoint},
  description={is the collection of allies the player has introduced.
  Once the \glsentrytext{pc} dies, the player takes their next \glsentrytext{pc} from the pool},
}

% Space

\longnewglossaryentry{region}{
  name={Regions},
  text={Region},
  prefix={a\space},
  type={mech},
  parent={sq},
  category={rules},
  nonumberlist,
  description={are broad types of areas, such as `Town', or `Forest'.

  Whenever the \glsfmtplural{pc} go from one Region to another, the \glsentrytext{gm} checks for the next available \glsentrytext{segment} in that Region (marked `\glsentrytext{sqr}')},
  }

% Actions

\longnewglossaryentry{action}{
  sort={T1.2},
  name={Actions},
  text={Action},
  type={mech},
  category={rules},
  prefix={an\space},
  description={When players want their \glsentrytext{pc} to attempt something risky, they roll $2D6$ plus \glsentrytext{attribute}, plus \glsentrytext{skill}},
}

\newacronym[
  description={means the number players need to roll on the dice to achieve a \emph{tie} with the task.
    Rolling higher indicates they have their prize, rolling lower means some nasty outcome is upon them,
    \noindent
    \begin{boxtable}
         \textbf{TN} & \textbf{Difficulty} \\
         \hline
         6 & Easy -- just ask the barmaid what you want. \\
         7 & Basic -- find firewood in the forest. \\
         10 & Tricky -- find a good price in the market. \\
         12 & Professional -- fix the cart by Sundown. \\
         14 & Specialist -- Plan a three-storey stone building. \\
    \end{boxtable}
  and rolling a tie means both (or neither)},
  shortplural={TNs},
  sort={S1},
  parent={action},
  prefix={a\space},
  ]{tn}{TN}{Tie Number}

\longnewglossaryentry{natural}{
  name={Natural Rolls},
  text={Natural Roll},
  sort={Natural Roll},
  prefix={a\space},
  sort={S6},
  parent={action},
  type={mech},
  category={rules},
  nonumberlist,
  description={represent the situation, and stay where they are; later rolls need to use the same result.

  If someone tries to figure out how to find their way out of the forest, and back to a road, the player could roll `\twoDice{4}'.
  With a +3 Bonus, the total is `7'.
  The next character has a +2 Bonus, so their total is `6'.
  With the \glsentrytext{tn} set at `10', the group cannot find their way back without changing their approach},
  }

\longnewglossaryentry{resistedaction}{
  name={Resisted Actions},
  text={Resisted Action},
  sort={S3},
  parent={tn},
  type={mech},
  category={rules},
  prefix={a\space},
  description={start at \glsentrytext{tn}~7, then add the \glsentrytext{npc}'s Bonuses.
    For example, a player declares their \glsentrytext{pc} wants to demand a new sword, but the \glsentrytext{gm} thinks the \glsentrytext{jotter} will just reflexively lie about supplies running low.

    The \glsentrytext{jotter}'s \roll{Wits}{Deceit} come to~+2 in total, so the \glsentrytext{tn} is ($7 + 2 =$) 9},
}

\longnewglossaryentry{restingaction}{
  name={Resting Actions},
  text={Resting Action},
  sort={S6},
  parent={action},
  type={mech},
  category={rules},
  prefix={a\space},
  description={apply when you can repeat something, without danger.
Set the darker die to `\dicef{6}' and roll the other.  If this roll fails, it fails forever},
}

\longnewglossaryentry{bandAct}{
  name={Banding Actions},
  sort={Banding Action},
  text={Banding Action},
  prefix={a\space},
  sort={S4},
  parent={action},
  type={mech},
  category={rules},
  nonumberlist,
  description={means characters perform better by working together.
  The first character adds their Bonus, the second adds half, the third, a quarter, et c. and we round halves up at the end},
  }

%% Travel

\longnewglossaryentry{travel}{
  name={Travel},
  text={travel},
  prefix={a\space},
  sort={T2},
  type={mech},
  category={rules},
  description={},
}

% Overland Journeys

\longnewglossaryentry{journey}{
  name={Journeys},
  text={journey},
  prefix={a\space},
  parent={travel},
  type={mech},
  category={rules},
  description={depends on the terrain.
    Good roads provide a smooth journey, rough roads provide less, and deep forests can demand attention and problem-solving twice a mile.

    Each \glsentrytext{interval}, the troupe can travel the initial distance easily.
    Every mile after that adds another~\glsentrytext{ep}.

    By default, characters might travel 10~miles per day, i.e. 5~miles in the morning, then 5~in the afternoon.
    By evening, most stop at the first \glsentrytext{bothy} they see, and take time to cook, and potentially meet another travelling group moving in the opposite direction.

    \marchingChart
    An expedient troupe might add 2~miles each \glsentrytext{interval}, and eat nothing but dry rations, so they can march instead of cooking in the evening.
    With 3~\glspl{interval} of movement, the troupe could travel 21~miles in total.
    The night would allow them to remove 1~\glsentrytext{ep}, but the day would add 6 more, so they won't be moving like that for long},
}

\longnewglossaryentry{vigil}{
  name={Vigils},
  text={vigil},
  first={vigil through the night},
  prefix={a\space},
  parent={travel},
  type={mech},
  category={rules},
  description={keep a troupe safe throughout dangerous nights outside, but inflict 2~\glsfmtplural{ep}.
  The players can divide these points among their characters as they wish},
}

\longnewglossaryentry{foraging}{
  name={Foraging},
  text={foraging},
  first={foraging for food},
  prefix={a\space},
  prefixfirst={while\space},
  parent={travel},
  type={mech},
  category={rules},
  description={can provide verdant supplies, easily, with just one \glsentrytext{interval} of foraging.
  The player rolls \roll{Intelligence}{Survival} at \tn[12] (+2 Bonus during \glsentrytext{cFive}, -2 Penalty over \glsentrytext{cTwo}, and a -4 Penalty at night).
  Success means enough basic ingredients for $1D6$ meals.

  However, preparing the meals requires an \roll{Intelligence}{Cultivation} roll at \tn[10]},
}

% Caving

\longnewglossaryentry{caving}{
  name={Caving},
  text={caving},
  prefix={a\space},
  parent={travel},
  type={mech},
  category={rules},
  description={means co\"ordinating the \glsentrytext{deep}, where long caverns spread like veins, the logistics of travel twist and invert, and the dangers change.

    The players decides how many miles to cover, then roll \roll{Dexterity}{Caving}.
    The \glsentrytext{tn} starts at 6, and each mile raises it by +2.

    \begin{boxtable}[YYY]
      \textbf{Miles} & \textbf{\glsentrytext{tn}} & \textbf{\Glsfmtplural{ep}} \\\hline
      0  & 6 & 0/ 1 \\
      1  & 8 & 1/ 2 \\
      2  & 10 & 2/ 3 \\
      3  & 12 & 3/ 4 \\
    \end{boxtable}
    On a tie, the entire group takes an extra \glsentrytext{ep} from skuffs and scrapes, and if the roll fails, the character walking at the front takes $1D6$~Damage from bashing their head, twisting their ankle, or falling down some hole},
}

\longnewglossaryentry{blackWalking}{
  name={Black Walking},
  text={black walking},
  prefix={a\space},
  parent={caving},
  type={mech},
  category={rules},
  description={means walking in the darkness.
  It adds +2 to the \glsentrytext{tn} to travel, so even travelling a short distance becomes dangerous},
}

\longnewglossaryentry{gagingCave}{
  name={Gaging Caverns},
  text={gaging the cavern},
  first={gaging a cavern},
  firstplural={gaging caverns},
  prefix={a\space},
  parent={caving},
  type={mech},
  category={rules},
  description={tells you the chance of nearby water, of cave-ins, and may even indicate precious metals.
  The player rolls \roll{Wits}{Caving} (\tn[10]) to avoid misunderstanding the signs in the dark or (\tn[9]) to spot potential cave-ins.
  Appropriate tools include chisels and light},
}

\longnewglossaryentry{echoing}{
  name={Echoing},
  text={echoing},
  first={cavern echoing},
  prefix={a\space},
  parent={caving},
  type={mech},
  category={rules},
  description={works like bat-sonar.
  Gnomes immitated these cries, and added claps, whistles, and different ways of craning the neck while waiting for a response.
  With practice (and a \roll{Wits}{Caving} roll), a caver can guess the size (\tn[10]) or shape (\tn[12]) of a tunnel, or even guess how a cavern develops over the next hundred metres (\tn[14])},
}

\longnewglossaryentry{hypoxia}{
  name={Hypoxia},
  text={hypoxia},
  prefix={a\space},
  parent={caving},
  type={mech},
  category={rules},
  description={means that air has grown thin, which makes people tired in a way they don't always notice.
  This happens in deep, narrow caverns, and becomes worse as more people breathe the same air, and much worse if any of them carry a flame.

  Anyone affected has the \glsentrytext{tn} for all actions increased (and the player should not be told), and will start to hallucinate (if the players theorize about something that might happen, they begin to hallucinate the exact thing they spoke about).
  Soon after, all fires go out, as the dead air suffocates them},
}

\longnewglossaryentry{caveFire}{
  name={Underground Fires},
  text={underground fire},
  prefix={an\space},
  parent={caving},
  type={mech},
  category={rules},
  description={demand a complete understanding of convection, air-pressure, and the type of fuel being used.
  The player rolls \roll{Intelligence}{Caving} (\tn[12]), to avoid filling the room with smoke.
  Inhaling the smoke inflicts 1~\glsentrytext{ep} each time},
}

%%%%% Combat

\longnewglossaryentry{combat}{
  sort={T1.4},
  name={Combat},
  text={Combat},
  type={mech},
  category={rules},
  prefix={a\space},
  description={It does not matter who initiates combat -- each character enters the standard \glsentrytext{resistedaction}.
  The \glsentrytext{pc} rolls \roll{Dexterity}{Melee}, and the \glsentrytext{tn} equals 7 + the \glsentrytext{npc}'s \roll{Dexterity}{Melee}.
  The winner deals $1D6+$ Damage + Strength Bonus, and every +4 Damage converts to $1D6$},
}

\longnewglossaryentry{retreat}{
  name={Retreat},
  text={retreat},
  prefix={a\space},
  parent={combat},
  type={mech},
  nonumberlist,
  description={works like any \glsentrytext{resistedaction}; both parties begin with \roll{Speed}{Athletics}.
  If either side wins with a Margin of 3 or more, they win (i.e. escape or capture).
  But if either rolls a lower Margin, both sides run through one \glsentrytext{area}, gain one \glsentrytext{ep}, and the winners can change the relevant \glsentrytext{skill} by deciding where or how they flee.

  For example, a troupe of characters could run through dense thickets so that both sides have to use \roll{Speed}{Survival} on the next roll; or in a town they might try to navigate through a dense crowd with \roll{Speed}{Empathy}.

  The \glsentrytext{gm} should give \glsentrytext{area}-options after a successful roll.
  Each roll inflicts 1~\glsentrytext{ep} on both sides},
  }

\longnewglossaryentry{area}{
  name={Areas},
  sort={Area},
  text={Area},
  prefix={an\space},
  parent={retreat},
  type={mech},
  nonumberlist,
  description={give a rough unit for large spaces.
  An area is a space made distinct by its features.
  In the \glsentrytext{deep}, each cavern might count as an area, while out in the open plains a forest might be composed of the local areas: `the centre with the big, felled tree', `the river's fork', and `the griffins' nesting site'},
  }

\newacronym[
  description={give a rough estimate of a creature's combat abilities, and the value of any \glsentrytext{monster}'s corpse},
  shortplural={CRs},
  prefix={a\space},
  name={Creature Ratings (CR)},
  nonumberlist,
  parent={combat},
  ]{cr}{CR}{Combat Rating}

\newacronym[
  description={provide a linear measure of a character's health or injuries},
  shortplural={HP},
  prefix={an\space},
  prefixfirst={a\space},
  symbol={\fullmoon},
  parent={combat},
  name={Health Points (HP)},
  ]{hp}{HP}{Health Point}

\longnewglossaryentry{round}{
  name={Rounds},
  text={round},
  prefix={a\space},
  parent={combat},
  type={mech},
  description={start when everyone wants to speak at once.
  The \glsentrytext{gm} goes round the table clockwise as players commit to actions by spending \glsfmtplural{ap}},
  }

\newacronym[
  description={measure how many actions someone can take in a round, based on how fast they can move and react.
  Start with 3 AP, plus your Speed; put that many coins on your character sheet, and spend them each time you take an action},
  shortplural={AP},
  name={Action Points (AP)},
  parent={combat},
  prefix={an\space},
  ]{ap}{AP}{Action Point}

\longnewglossaryentry{quickaction}{
  name={A Response Action},
  text={Response Action},
  sort={Response},
  prefix={a\space},
  parent={ap},
  type={mech},
  category={rules},
  nonumberlist,
  description={means the character must resist some \glsentrytext{resistedaction}.
  If the \glsentrytext{ap} loss push them below 0, then every negative becomes a penalty to all action},
}

\longnewglossaryentry{running}{
  name={Running},
  text={running},
  prefix={a\space},
  parent={ap},
  type={mech},
  category={rules},
  nonumberlist,
  description={costs 1 \glsentrytext{ap}, and lets the character move three \glsfmtplural{step} plus their Athletics},
}

\longnewglossaryentry{step}{
  name={Steps},
  text={step},
  prefix={a\space},
  parent={ap},
  type={mech},
  nonumberlist,
  description={provide a rough measure of space.
  We can imagine it about a metre long, or as wide as the step on your gaming board, or any other length},
}

\newacronym[
  description={measure how much luck the character has left.
  Spend them to avoid Damage.
  Your maximum $\Glsfmtplural{fp} = \frac{Total~\glsfmtplural{xp}}{10} + Charisma$},
  shortplural={FP},
  prefix={an\space},
  prefixfirst={a\space},
  symbol={\adfdiamond},
  parent={combat},
  name={Fate Points (FP)},
  ]{fp}{FP}{Fate Point}

\longnewglossaryentry{weapon}{
  name={Weapons},
  text={weapon},
  type={mech},
  prefix={a\space},
  nonumberlist,
  parent={combat},
  description={add to Attack and Damage.
  Smaller weapons only cost 1~\glsentrytext{ap} to use, while larger weapons cost more, but also have bigger Bonuses},
}

\longnewglossaryentry{ambush}{
  name={Ambushes},
  text={ambush},
  plural={ambushes},
  type={mech},
  prefix={an\space},
  nonumberlist,
  parent={combat},
  description={grant one extra \glsentrytext{ap} for each Margin rolled while planning.
  The roll is \roll{Intelligence}{Combat}, against the opponent's \roll{Wits}{Combat}},
}

\longnewglossaryentry{armour}{
  name={Armour},
  text={armour},
  type={mech},
  prefix={an\space},
  nonumberlist,
  parent={combat},
  description={protects characters by reducing Damage},
}

\newacronym[
  description={reduces incoming Damage, before a single \glsentrytext{fp} is spent.
  It usually represents armour},
  shortplural={DR},
  parent={armour},
  prefix={a\space},
  ]{dr}{DR}{Damage Resistance}

\longnewglossaryentry{covering}{
  name={Covering},
  type={mech},
  category={rules},
  prefix={a\space},
  nonumberlist,
  parent={armour},
  description={means how much \glsentrytext{armour} covers the body.
  \Glsentrytext{armour} with `Covering 3' protects the torso and may have a helmet, while armour with `Covering 5' protects almost the entire body},
}

\longnewglossaryentry{vitalShot}{
  name={Vitals Shots},
  text={Vitals Shot},
  sort={Vitals Shot},
  prefix={a\space},
  nonumberlist,
  type={mech},
  category={rules},
  parent={armour},
  description={are attacks which equal a target's \glsentrytext{tn} plus their \glsentrytext{armour}'s \glsentrytext{covering}; this lets the attack ignore the \glsentrytext{armour}'s \glsentrytext{dr}, and deal direct Damage.

  If a player needs to roll at \glsentrytext{tn}~10 to hit an opponent with `\glsentrytext{covering}~3', then they need to roll `13' to make a Vitals Shot.
  This applies symmetrically; if the \glsentrytext{pc}'s armour has `\glsentrytext{covering}~5', and they miss by 5, then their opponent scores a Vitals Shot, and their armour counts for nothing, providing no \glsentrytext{dr}},
  }

\longnewglossaryentry{swarm}{
  name={Swarms},
  text={swarm},
  type={mech},
  category={rules},
  symbol={\Juno},
  prefix={a\space},
  nonumberlist,
  parent={combat},
  description={
    are myriad tiny creatures, acting as one.
    They crawl over characters, and into gaps in armour.

    Swarms can cover a number of \glspl{step} equal to their \glsentrytext{hp}-total, or bunch up together, with 3~\glsfmtplural{hp} per \glsentrytext{step}.

    Attacking swarms is easy when there are so many targets.
    The \glsentrytext{tn} to attack always reduces by 1 per \glsentrytext{hp} in the swarm, so when a swarm is listed with `{\scshape Att 12 - 8 \glsentrytext{hp}}', the \glsentrytext{tn} would be only 4; but if the swarm had only 1~\glsentrytext{hp} left, hitting it would require a roll at \glsentrytext{tn}~11.
    However, swarms only take 1 Damage each per attack.

    Swarms can split into smaller parts as a normal movement action.
    Each part inflicts 1~Damage each \glsentrytext{round} to anyone on the same \glsentrytext{step}, as long as the swarm's \glsentrytext{hp} total comes to more than the target's \glsentrytext{covering}},
}

\longnewglossaryentry{projectiles}{
  name={Projectiles},
  text={Projectiles},
  plural={Projectiles},
  type={mech},
  category={rules},
  prefix={a\space},
  parent={combat},
  description={rolls use \roll{Dexterity}{Projectiles}, and targets resist with \roll{Speed}{Vigilance}.
    Every 5 \glsfmtplural{step}' distance adds +1 to the \glsentrytext{tn}.
    When \glsfmtplural{pc} hit the \glsentrytext{tn} precisely, they miss their first target, but hit another target behind (if any)},
}

\longnewglossaryentry{crossbow}{
  name={Crossbows},
  text={crossbow},
  type={mech},
  prefix={a\space},
  parent={projectiles},
  description={only need 1~\glsentrytext{ap} to fire.
  They grant a Bonus to hit equal to 4 -\glsentrytext{weight} and deal Damage equal to their \glsentrytext{weight} -2, doubled.

  Reloading a crossbow requires 5~rounds, plus the weapon's Damage, and the user must have a Strength Bonus at least as high as the weapon's \glsentrytext{weight}},
}

\longnewglossaryentry{bow}{
  name={Hunting Bows},
  text={hunting bow},
  type={mech},
  prefix={a\space},
  parent={projectiles},
  description={deal any amount of Damage, depending on the bow, but cannot be pulled back by someone with a Strength Bonus lower than the Damage.
  The \glsentrytext{ap}~cost to pull one back equals 2 plus its Damage.

  The hunting bow gives a Bonus to hit equal to its Damage, if the archer has time to draw properly (i.e. they still have at least 1~\glsentrytext{ap} after firing).
  Flustered archers, take the weapon's Bonus as a penalty if they would not be able to fire in time},
}

\longnewglossaryentry{impromptuThrownWeapons}{
  name={Impromptu Thrown Weapons},
  text={impromptu thrown weapon},
  type={mech},
  prefix={an\space},
  parent={projectiles},
  description={receive a -2 penalty to hit and Damage, and a further -1~Penalty per \glsentrytext{step}~thrown},
}

% Equipment

\longnewglossaryentry{equipment}{
  sort={T1.7},
  name={Equipment},
  text={equipment},
  prefix={an\space},
  type={mech},
  category={rules},
  nonumberlist,
  description={Items can be held in a hand, or in a backpack.
  By default, each provides a Bonus equal to its \glsentrytext{weight}, but various items buck the trend one way or another},
}

\longnewglossaryentry{weight}{
  name={Weight Rating},
  text={Weight},
  first={Weight Rating},
  prefix={a\space},
  symbol={\astrosun},
  sort={A},
  type={mech},
  nonumberlist,
  parent={equipment},
  description={Characters can carry items with a total Weight Rating equal to their \glsentrytext{hp} total.
  Each point beyond inflicts a -1 Penalty to all actions.
  Creatures have a \glsentrytext{weight} equal to their own \glsentrytext{hp}},
}

\newacronym[
  description={measure how tired, hungry, and fed-up characters feel.
  Each EP has a \glsentrytext{weight} of~1},
  shortplural={EP},
  prefix={an\space},
  parent={weight},
  name={Exhaustion Points (EP)},
  ]{ep}{EP}{Exhaustion Point}

\longnewglossaryentry{coin}{
  name={Coinage},
  text={coin},
  prefix={a\space},
  sort={Y},
  type={mech},
  nonumberlist,
  parent={equipment},
  description={can become heavy quickly, gaining a total \glsentrytext{weight} of 1 for every 100, so a small chest of 1,000 coins would have a total \glsentrytext{weight} of 10},
}

\newacronym[
  description={are the smallest unit of currency},
  shortplural={cp},
  name={Copper Pieces ({\scshape cp})},
  type={mech},
  category={rules},
  longplural={copper pieces},
  prefix={a\space},
  parent={coin},
]{cp}{cp}{copper piece}

\newacronym[
  description={gets you 100 copper pieces},
  name={Silver Pieces ({\scshape sp})},
  sort={Y},
  type={mech},
  category={rules},
  longplural={silver pieces},
  shortplural={sp},
  prefix={a\space},
  parent={coin},
  ]{sp}{sp}{silver piece}

\newacronym[
  description={convert to ten silver, or a thousand copper pieces},
  name={Gold Pieces ({\scshape gp})},
  shortplural={gp},
  longplural={gold pieces},
  sort={Z},
  type={mech},
  category={rules},
  prefix={a\space},
  parent={coin},
]{gp}{gp}{gold piece}


% Magic

\longnewglossaryentry{witchcraft}{
  name={Witchcraft},
  text={witchcraft},
  sort={T2.1},
  prefix={a\space},
  type={mech},
  description={Some speak their \glsentrytext{spell}, others construct it from \glsentrytext{monster} bodies.
  In any case, the results are the same -- unpredictable},
}

\newacronym[
  name={Mana Points (MP)},
  shortplural={MP},
  prefix={an\space},
  prefixfirst={a\space},
  symbol={\sun},
  parent={witchcraft},
  description={grant every \glsentrytext{witch} their power.
  When they run out, they gain one \glsentrytext{ep} for every point they cannot spend},
  ]{mp}{MP}{Mana Point}

\longnewglossaryentry{spell}{
  name={Spells},
  text={spell},
  prefix={a\space},
  parent={witchcraft},
  type={mech},
  category={rules},
  description={have a mind of their own.
  Once cast, they endure until they burn through themselves, or something destroys them.
  To stop a Fire spell, someone must put the fire out, and if an angry \glsentrytext{witch} makes antlers grow on someone's head, the only way to `dispel' them is with a boning knife.

  Spells with a Distant range cannot be reigned in; if the range is
  \toggletrue{Distant}%
  \setcounter{spellCost}{4}\setRange%
  `\spellRange', the spell will find the nearest target at that distance.

  Casters only select a spell's first target.
  The spell forks through the others like lightning, and may `arc' across any distances up to its original range.
  Water spells which hit a river will spread through the river, but a curse with an `area' of 4 will have to jump until it has found four people to inflict itself on},
}

\longnewglossaryentry{invocation}{
  name={Invocations},
  text={Invocation},
  prefix={an\space},
  parent={witchcraft},
  type={mech},
  category={rules},
  description={are the basic sentence-formulae which define spells.
    They consist of one to five `descriptors', one action, and a target.
    \noindent
    \begin{boxtable}[XccL]
         \textbf{Descriptors} & \textbf{Action} & \textbf{Target} & \textbf{Result} \\
         \hline
         ---                  & Wax             & Fire            & Candle Grows Bright   \\
         Detailed             & Warp            & Water           & Water turns into an ice statue. \\
         Distant, Duplicated  & Wane            & Fate, Air       & Targets at a distance ignore \glsentrytext{ep} Penalties. \\
    \end{boxtable}
    If a mechanical effect needs a number, that number is 2 when using an elemental \glsentrytext{sphere}, and 1 when using a high \glsentrytext{sphere}.
    Each descriptor raises the \glsentrytext{mp} cost and mechanical effects by 1},
}

\longnewglossaryentry{witch}{
  name={Witch},
  text={witch},
  plural={witches},
  prefix={a\space},
  type={mech},
  parent={witchcraft},
  description={simply means anyone who can natural speak to an elemental \glsentrytext{sphere} using their inner \glsentrytext{mp} store.
  These people have no special uniform, and often hide their talents},
}

\longnewglossaryentry{casting}{
  name={Castings},
  text={casting},
  prefix={a\space},
  parent={witch},
  type={mech},
  category={rules},
  description={start by spending one \glsentrytext{mp} per spell level.
  The \glsentrytext{witch} then commands the target \glsentrytext{sphere}, rolling \roll{Charisma}{} the lowest \glsentrytext{skill} required.

  When `overspending' on the \glsentrytext{invocation}, the debt is paid in \glsentrytext{ep}.

  \Glsfmtplural{tn} depend on how malleable the target is.
  Earth spells can affect ice far more easily than rocks, and Air spells can whip up a gale easier when outdoors},
}

\longnewglossaryentry{alchemy}{
  name={Alchemy},
  text={alchemy},
  sort={Alchemy},
  symbol={\glsentrytext{R}},
  prefix={an\space},
  type={mech},
  parent={witchcraft},
  description={is the practice of turning a raw magical \glsentrytext{ingredient} into something useful.
  It requires no \glsentrytext{sphere} \glsentrytext{skill} to use -- just a recipe},
}

\longnewglossaryentry{sphere}{
  name={Spheres},
  text={Sphere},
  prefix={a\space},
  parent={witchcraft},
  type={mech},
  description={divide the world into meaningful parts.
  The five elemental Spheres are Fire, Air, Fate, Water, and Earth.
  Each one can join with two neighbours to make one of the high Spheres; Light, Death, Mind, Life, and Force.

  If a caster can think of a way to use any Sphere to stop an attack, they can enter combat as usual with their \roll{Charisma}{Sphere}, rolling at \glsentrytext{tn} 7 plus the \glsentrytext{npc}'s \roll{Dexterity}{Melee}.
  A battle-ready \glsentrytext{witch} might encourage a warrior's torch to burn his own face off, or make him forget what he wanted to do a moment before his sword comes down},
}

\longnewglossaryentry{ingredient}{
  name={Ingredients},
  text={Ingredient},
  prefix={an\space},
  nonumberlist,
  category={indexed},
  parent={alchemy},
  type={mech},
  description={are the basic materials used to make any \glsentrytext{boon}, or \glsentrytext{talisman}, and for lots of medicines.
  Each has an elemental affinity, so a Fire Ingredient can only make a Fire \glsentrytext{boon}},
}

\longnewglossaryentry{boon}{
  name={Concoctions},
  text={Concoction},
  prefix={a\space},
  sort={concoction},
  parent={alchemy},
  category={indexed},
  type={mech},
  description={are liquids or powders which, when thrown in the air, hyper-charge the use of a single magic \glsentrytext{sphere}, for anyone present next to the burst.
  For example, a concoction to boost the Air \glsentrytext{sphere} would mean a caster with Air 2 could cast a single spell as if they had Air 3.
  Using one in combat requires at least one \glsentrytext{ap} to grab it, and another to dispurse it into the air},
}

\longnewglossaryentry{elixir}{
  name={Elixirs},
  text={elixir},
  prefix={an\space},
  parent={alchemy},
  category={indexed},
  type={mech},
  description={heal diseases.
  Each one requires a particular type of \glsentrytext{ingredient} to heal a particular disease},
}

\longnewglossaryentry{talisman}{
  prefix={a\space},
  name={Talismans},
  text={Talisman},
  first={talisman (a one-use alchemical item)},
  firstplural={talismans (one-use alchemical items)},
  sort={Talisman},
  category={indexed},
  type={mech},
  parent={alchemy},
  description={are spells, locked in an item, along with some activation condition.
  A talisman could open a magical gateway once it reaches a certain location, or bless the first person it sees with good luck.
  Many will strike the nearest, available target once activated, which makes them dangerous in the wrong hands},
}

\longnewglossaryentry{artefact}{
  name={Artefacts},
  text={Artefact},
  prefix={an\space},
  type={mech},
  parent={alchemy},
  description={happen, often by accident, when someone imbues sentience into an unused \glsentrytext{talisman}, then leaves it to contemplate its existence for a century.
  spells given sentience, and function as long-term magical items.
  They frequently go awry, as they have a mind of their own, and their own wishes and values},
}



%%%%%%%%%%%%%%%% General Terms %%%%%%%%%%%%%%%%%%%%

\longnewglossaryentry{cosmology}{
  name={Cosmology},
  text={cosmology},
  nonumberlist,
  description={every year, the \glsentrytext{ainumar} orbits the Sun, and every \glsentrytext{cycle}, we orbit the \glsentrytext{ainumar}},
}

\longnewglossaryentry{ainumar}{
  name={The Ainumar},
  text={Ainumar},
  plural={Ainumari},
  parent={cosmology},
  nonumberlist,
  description={shines brightly, in our sky.
  At the end of each \glsentrytext{cycle}, it grows, massive, and you can see a storm raging across its face.
  Many think that the gods live there, planning how to kill people, and take their souls up to their houses.

  Each god holds domain over a different death},
}

\longnewglossaryentry{fenestra}{
  name={Fenestra},
  nonumberlist,
  sort={3},
  description={This land, where elves, gnolls, and humans look up at trees, like ants moving through blades of grass.
  Predators larger than a horse hunt deer and people in the same way, so everyone travels together, and well-armed},
}

\newacronym[
  description={is the universal way to measure time in \glsentrytext{fenestra}, where one \glsentrytext{cycle} equals sixty days, and, and one sixth of a year},
  name={Gnomish Machine Time (GMT)},
  prefix={a\space},
  parent={fenestra},
  type={main},
]{gmt}{GMT}{Gnomish Machine Time}

\longnewglossaryentry{cycle}{
  name={Cycles},
  text={cycle},
  sort={Z},
  prefix={a\space},
  parent={cosmology},
  description={last for sixty days, after which \glsentrytext{fenestra} has travelled around the \glsentrytext{ainumar}.
  Each cycle begins and ends with a violent \glsentrytext{storm}, which marks a change in temperature for the next cycle.

  \orrery

  After six cycles, the \glsentrytext{ainumar} completes a revolution around the Sun, and a new year begins},
}

\longnewglossaryentry{edge}{
  name={The Edge},
  sort={Edge},
  text={Edge},
  first={Edge of Civilization},
  prefix={an\space},
  category={location},
  parent={fenestra},
  description={lies one footstep off the \glsentrytext{lonelyRoad}, and surrounds every outer \glsentrytext{village}.
  Beyond this point, only dark forests, empty tundra, and hungry beasts wait.
  When people travel off-road, they have gone beyond the Edge},
}

\longnewglossaryentry{lonelyRoad}{
  name={The Lonely Road},
  sort={Lonely Road},
  text={lonely road},
  prefix={a\space},
  category={location},
  parent={fenestra},
  description={means any road between settlements.
  Going from one town to the next means a long journey through untamed territory},
}

\longnewglossaryentry{lonelyTavern}{
  name={Lonely Taverns},
  text={lonely tavern},
  prefix={a\space},
  category={location},
  parent={lonelyRoad},
  description={exist on long stretches of empty road, where enough traders pass through to create a miniature settlement, sustained only by the goods people bring.
  They make their own ale, and keep their own laws},
}

\longnewglossaryentry{tradeTongue}{
  name={The Trade Tongue},
  text={Trade Tongue},
  sort={Trade Tongue},
  category={indexed},
  parent={lonelyRoad},
  description={lets people trade, despite not sharing much of a common language.
  It has about two hundred words, so people have to indicate what they want to say by lumping words together.
  E.g. `cow' might be `white-water animal', and cheese could be `rock of white-water animal'},
}

\longnewglossaryentry{deep}{
  name={The Labyrinth},
  text={Labyrinth},
  sort={Labyrinth},
  prefix={a\space},
  category={location},
  parent={fenestra},
  description={is the network of frigid, nearly lifeless caverns, which sits beneath much of \glsentrytext{fenestra}},
}

\longnewglossaryentry{blight}{
  name={A Blight},
  text={blight},
  sort={Blight},
  prefix={a\space},
  category={location},
  parent={fenestra},
  description={is an area too full of goblins for anyone to live.
  Most blights have at least one cave or portal to the underground goblin realm},
}

\longnewglossaryentry{village}{
  name={Baileys},
  text={bailey},
  prefix={a\space},
  category={location},
  parent={fenestra},
  description={are walled villages, which stands beyond the protection of any towns, and endure attacks by wandering monsters.
  They mark the \glsentrytext{edge} of civilization, as nothing lies beyond them except the wild forest.

  A standard bailey's walls stretch at least 50 steps in diameter.
  Beyond that, the farmland stretches out another 100 steps; for a minimum diameter of 250 steps.
  The outer perimiter pushes the forest back another 50 steps, with more for a healthy bailey, and a smaller safety zone in less well-tended baileys.

  If the people living in the bailey lack the strength and skill to keep the forest back, the safety perimiter grows smaller as bushes creep towards the farmland, providing cover for monsters, and diminishing the time archers have to shoot at creatures},
}

\longnewglossaryentry{bothy}{
  name={Bothies},
  text={bothy},
  plural={bothies},
  prefix={a\space},
  category={location},
  parent={lonelyRoad},
  description={are small half-way houses on long roads, built so that travellers can sleep safely after Sundown.
  Some have a single fireplace, and enough room for a half a dozen people and a donkey},
}

\longnewglossaryentry{fiend}{
  name={Fiends},
  text={fiend},
  prefix={a\space},
  parent={fenestra},
  category={indexed},
  description={live outside any civilization, which makes them the enemy.
  Many know magic, others hold armies.
  They make their own laws, in their own realms, and most leave them alone.
  When civilization prods them too much, some have been known to destroy entire cities and every \glsentrytext{village} around},
}

\longnewglossaryentry{hag}{
  name={Hags},
  text={hag},
  prefix={a\space},
  parent={fiend},
  description={are old ladies with too much spite, plans, and magic to die.
    With a few spells here and there to increase their life span, and another to hibernate for a couple of decades while their plans mature, they extend themselves a couple of centuries beyond their rightful lifespans.
    They all die in the end, but hags mostly die through violence},
}

\longnewglossaryentry{bandit}{
  name={Bandits},
  text={bandit},
  prefix={a\space},
  parent={fiend},
  description={begin when farmers run out of food, or city-dwellers lose their jobs, they have only two option -- banditry or the \glsentrytext{guard}.
  People generally select whichever comes along to recruit them first},
}

\longnewglossaryentry{lich}{
  name={Liches},
  text={lich},
  plural={liches},
  prefix={a\space},
  parent={fiend},
  description={begin as students of Death magic.
    They can slow, and eventually stop, aging, which locks them into a permanent state of semi-death, and leaves them unable to regenerate a single \glsentrytext{mp} without draining it from someone's death},
}

\longnewglossaryentry{dryad}{
  name={Dryads},
  text={dryad},
  prefix={a\space},
  parent={fiend},
  description={are old elves, but older elves get \textit{weird}.
  They experiment with their bodies, adopting animal parts, and soon lose interest in socialising with other elves.
  Most dryads are harmless most of the time, taking more intest in a local flower than anything else around them.
  However, once they get sufficiently disconnected from normal people, they stop seeing people as people, and start to see them as animals.

  And like any elf, dryads eat animals},
}

\longnewglossaryentry{dragon}{
  name={Dragons},
  text={dragon},
  prefix={a\space},
  parent={fiend},
  description={\ldots everyone knows what a dragon is},
}

\longnewglossaryentry{ogre}{
  name={Ogres},
  text={ogre},
  prefix={an\space},
  parent={fiend},
  description={begin as goblins, but goblin height is limited only by food.
  Once they eat enough, they grow and grow, until starvation beckons -- they cannot reduce their intake after growing too much.
  Grumpy ogres die at this point.
  More sociable, or at least tactical ogres, survive by leading bands of goblins},
}

\longnewglossaryentry{monster}{
  name={Monsters},
  text={monster},
  prefix={a\space},
  parent={fenestra},
  symbol={\glssymbol{sylf}},
  description={wander slowly, looking for deer, auroch, or anything they can eat.
  When they hear people, the noises and lights excite them, and they instinctively begin to stalk},
}

\longnewglossaryentry{woodspy}{
  name={Woodspies},
  text={woodspy},
  plural={woodspies},
  prefix={a\space},
  parent={monster},
  description={are a kind of land-octopus which can camouflage itself perfectly, changing the colour and tecture of their skin.
  They are highly intelligent, but do not understand communication beyond what they need to mate},
}

\longnewglossaryentry{digger}{
  name={Mouthdiggers},
  text={mouthdigger},
  prefix={a\space},
  parent={monster},
  description={look like star-nosed moles about the size of a wolf, with extensible jaws, which stretch out to also be the size of a wolf.
  They rely entirely on ambushes, in which they drag their quarry back into their set by sinking their teeth into a leg and retreating underground},
}

\longnewglossaryentry{crawler}{
  name={Chitincrawlers},
  text={chitincrawler},
  prefix={a\space},
  parent={monster},
  description={are arachnids about the size of a horse.
  Some say the face looks like a wolf's, others say it looks almost human.
  Either way, the apparent smile is just a shape, as they have no ability to feel or think, they only have hunger, and sharp claws at the end of the front legs},
}

\longnewglossaryentry{basilisk}{
  name={Basilisks},
  text={basilisk},
  prefix={a\space},
  parent={monster},
  description={have snake-like bodies, with six arms.
  Some grow to the size of a caravan.

  They move slowly, always conserving energy for the hunt.
  But when hunting, they can charge, and let out a cloud of disgusting breath, making their prey wretch},
}

% Guilds

\longnewglossaryentry{temple}{
  name={The Divine Guilds},
  sort={Divine Guilds},
  text={temple},
  sort={6},
  prefix={a\space},
  nonumberlist,
  description={Each temple exists to protect people from a god, by selling goods or services.
  Each temple functions as a guild, with a divine monopoly on their protection},
}

\longnewglossaryentry{sylf}{
  name={Sylf},
  category={god},
  symbol={\Aries},
  parent={ainumar},
  description={has griffin wings, with a woodspy's head, an arachnid thorax, and human belly; both chambers are painfully bloated from pregnancy.
  She gives birth to monsters endlessly, and they eat someone, she uses their soul to birth a new creature},
}

\longnewglossaryentry{templeOfBeasts}{
  name={The Temple of Beasts},
  text={Temple of Beasts},
  plural={Temples of Beasts},
  prefix={a\space},
  symbol={\glsentrysymbol{sylf}},
  parent={temple},
  nonumberlist,
  description={The highest and lowest of all temples absorbs feckless drunks, dickheads, scum, and people with political progressive political ideas.
  All of them become heroes, and forest-feed},
}

\longnewglossaryentry{broch}{
  name={Brochs},
  text={broch},
  prefix={a\space},
  category={location},
  sort={ZZ},
  parent={templeOfBeasts},
  description={are grand towers which surround civilization.
  The \glsentrytext{guard} stay in them, playing loud pipes, and lighting fires to attract monsters.
  A ring of flat earth surrounds each broch, giving archers a clear shot at anything which emerges from the \glsentrytext{edge}.
  The noise echoes up to five miles around, and by dusk, the archers stand ready.

  Sometimes a monster dies, most flee -- still alive, but with a painful lesson about approaching the sound of people, pipes, and song.
  Each predator which comes to a broch means one less attack upon some inner \glsentrytext{village}, or on the \glsentrytext{lonelyRoad}.

  Each broch takes charge of a food-producing \glsentrytext{village} or two, which sustains the \glsentrytext{guard}},
}

\longnewglossaryentry{guard}{
  name={Night Guards},
  sort={Night Guard},
  text={Night Guard},
  prefix={a\space},
  parent={templeOfBeasts},
  category={profession},
  sort={3},
  nonumberlist,
  description={are the sorry lot who have nothing better to do than wander into the darkness and get eaten.
  They exist to either thwart, or feed, \glsentrytext{sylf}, depending on whom one asks},
}

\longnewglossaryentry{fodder}{
  name={Fodder},
  text={fodder},
  plural={fodder},
  prefix={some\space},
  sort={R0},
  parent={templeOfBeasts},
  category={rank},
  description={are the lowest rung of the \glsentrytext{guard}.
  Most who enter as fodder arrived as criminals, vagrants, or political idealists.
  About half of these new recruits will survive and gain the next rank.

  Each of these criminals begins with a 100 \glsentrytext{sp} debt to repay to their temple},
}

\longnewglossaryentry{gDigger}{
  name={Grave Diggers},
  text={digger},
  prefix={a\space},
  sort={R1},
  parent={templeOfBeasts},
  category={rank},
  description={have survived a mission, and returned alive.
  In theory, they should burry the fodder who did not survive that mission, but in practice, few people leave a body},
}

\longnewglossaryentry{soldier}{
  name={Archers},
  text={archer},
  prefix={an\space},
  sort={R2},
  parent={templeOfBeasts},
  category={rank},
  description={stand on \glsentrytext{broch} balconies, taking down summoned by the pipes.

  Farmers who guard their \glsentrytext{village} can sign up to the \glsentrytext{guard} and begin at this rank immediately},
}

\longnewglossaryentry{cutter}{
  name={Cutters},
  text={cutter},
  prefix={a\space},
  sort={R3},
  parent={templeOfBeasts},
  category={rank},
  description={guard caravants along the long, \glsentrytext{lonelyRoad}, or take special missions, beyond the \glsentrytext{edge}},
}

\longnewglossaryentry{ranger}{
  name={Rangers},
  text={ranger},
  prefix={a\space},
  sort={R4},
  parent={templeOfBeasts},
  category={rank},
  description={travel fast, often on horseback, to provide reinforcements to any \glsentrytext{village} in immediate trouble.
  They travel twice as fast if anyone kills a forest \glsentrytext{monster} without giving the blessings of the \glsentrytext{templeOfBeasts}},
}

\longnewglossaryentry{jotter}{
  name={Jotters},
  text={jotter},
  prefix={a\space},
  sort={R5},
  parent={templeOfBeasts},
  category={rank},
  description={do paperwork for the \glsentrytext{guard}, and control everything that their seniors don't care to manage},
}

\longnewglossaryentry{thane}{
  name={Thanes},
  text={thane},
  prefix={a\space},
  sort={R6},
  parent={templeOfBeasts},
  category={rank},
  description={have risen to the point where any \glsentrytext{jotter} will finally leave them alone.
  Most try to find some gold and retire at this point},
}

\longnewglossaryentry{builder}{
  name={Builders},
  text={builder},
  prefix={a\space},
  sort={R7},
  parent={templeOfBeasts},
  category={rank},
  description={organize new settlements, which requires an intrepid \glsentrytext{doula} to help locating a good spot.
  It usually involves a lunch with a friendly, local, \glsentrytext{warden}},
}

\longnewglossaryentry{overseer}{
  name={Reeves},
  text={reeve},
  prefix={a\space},
  sort={R8},
  parent={templeOfBeasts},
  category={rank},
  description={organize the \glsentrytext{guard}, and have dinner with any local \glsentrytext{warden} who'll have them},
}

\longnewglossaryentry{abderian}{
  name={Abderian},
  category={god},
  symbol={$\Large\Circle$\hspace{-1em}\adfflowerleft},
  parent={ainumar},
  description={is the goddess of poison and rot.
  When she kills someone, she brings them to her banquet of pain, to see how long they can resist eating her rancid food},
}

\longnewglossaryentry{templeOfPoison}{
  name={The Temple of Poison},
  text={Temple of Poison},
  plural={Temples of Poison},
  prefix={a\space},
  symbol={\glsentrysymbol{abderian}},
  parent={temple},
  nonumberlist,
  description={Beer, brewing and baking protect humanity from \glsentrytext{abderian}, so this temple have established a divine monopoloy on the lot},
}

\longnewglossaryentry{wheatGuild}{
  name={The Wheat Guild},
  text={Wheat Guild},
  prefix={a\space},
  category={location},
  sort={ZZ},
  parent={templeOfPoison},
  description={is where drinks are brewed, and drunks can drink.
  An image of a skeleton, tempting you to feast with her, reminds patrons why they should always eat at an official Wheat Guild hall},
}

\longnewglossaryentry{server}{
  name={Servers},
  text={Server},
  parent={templeOfPoison},
  category={profession},
  sort={3},
  prefix={a\space},
  description={make food for the \glsentrytext{wheatGuild} all day, and every \glsentrytext{warden} has one in their employ},
}

\longnewglossaryentry{eldren}{
  name={Eldren},
  category={god},
  symbol={\Sun},
  parent={ainumar},
  description={takes those who die of sickness or age.
  Many people save up their whole lives to be allowed into the \glsentrytext{healersGuild} once they become old, so they can die in peace, and go to his realm},
}

\longnewglossaryentry{templeOfSickness}{
  name={The Temple of Sickness},
  text={Temple of Sickness},
  plural={Temples of Sickness},
  prefix={a\space},
  symbol={\glsentrysymbol{eldren}},
  parent={temple},
  nonumberlist,
  description={People only like one god.
  They all want \glsentrytext{eldren} to take them, rather than any alternatives, so anyone who has a little to spare puts it towards this temple, and hopes to see a bed inside one day, and die peacefully},
}

\longnewglossaryentry{healersGuild}{
  name={Healers' Guilds},
  prefix={a\space},
  category={location},
  sort={ZZ},
  parent={templeOfSickness},
  description={take in the sick, disabled, or dying of \glsentrytext{fenestra}, where they tend to each other.
  Long-term staff always have some long-term disability, such as missing limbs.
  Many of the \glsentrytext{guard} start a new career here},
}

\longnewglossaryentry{helper}{
  name={Helpers},
  text={Helper},
  prefix={a\space},
  parent={templeOfSickness},
  category={profession},
  sort={3},
  nonumberlist,
  description={tend to the sick and dying, on behalf of the \glsentrytext{healersGuild}.
  The temple only takes on new people from those with a disability of some kind, so the grounds are maximally accessible to everyone.
  The \glsentrytext{guard} who endure permanent injuries often retire as Helpers},
}

\longnewglossaryentry{mixer}{
  name={Mixers},
  text={mixer},
  prefix={a\space},
  parent={templeOfSickness},
  category={rank},
  description={can devise \glsentrytext{elixir} recipes to cure diseases.
  They create a constant demand on the \glsentrytext{ingredient}-trade, as people are always getting sick with one thing or another},
}

\longnewglossaryentry{counter}{
  name={Counters},
  text={counter},
  prefix={a\space},
  parent={templeOfSickness},
  category={rank},
  description={count the spare beds, pension-funds paid towards the \glsentrytext{templeOfSickness}, expected time until a sick person dies, total \glsentrytext{elixir}-count of each type, and expected total cures and corpses.
  They form the heart of every \glsentrytext{healersGuild}},
}

\longnewglossaryentry{grummel}{
  name={Grummel},
  category={god},
  symbol={\lightning},
  parent={ainumar},
  description={waits till \glsentrytext{fenestra} approaches the \glsentrytext{ainumar} and begins to shake it, causing earthquakes, volcanic erruptions, and tidal waves.
  These mark the end of each \glsentrytext{cycle}, and people call them Grummel's \glsentrytext{storm}},
}

\longnewglossaryentry{templeOfStorms}{
  name={The Temple of Storms},
  text={Temple of Storms},
  plural={Temples of Storm},
  prefix={a\space},
  symbol={\glsentrysymbol{grummel}},
  parent={temple},
  nonumberlist,
  description={sell calendars and architectural plans.
  The former help farmers, while the latter help any \glsentrytext{warden} who wants their dwelling to survive the next \glsentrytext{storm}},
}

\longnewglossaryentry{nulla}{
  name={\qquad},
  category={god},
  prefix={a\space},
  parent={ainumar},
  symbol={$\Large\Circle$},
  type={peeps},
  description={\qquad},
}

\longnewglossaryentry{templeOfMisgenesis}{
  name={The Temple of Misgenesis},
  text={Temple of Misgenesis},
  plural={Temples of Misgenesis},
  prefix={a\space},
  symbol={\glsentrysymbol{nulla}},
  parent={temple},
  nonumberlist,
  description={Some things never were.  This loose organization alters fortune to ensure nobody has to fail a task before they try, or give up on the best career they never considered},
}

\longnewglossaryentry{doula}{
  name={Doulas},
  text={Doula},
  prefix={a\space},
  parent={templeOfMisgenesis},
  category={profession},
  sort={3},
  description={help with births, blessings, and beginnings of all kind.
  They protect the population from misgenesis -- the death which occurs before life.
  Nobody begins a business venture or party without a blessing from a Doula},
}

\longnewglossaryentry{wayfinder}{
  name={Wayfinders},
  text={wayfinder},
  prefix={a\space},
  parent={templeOfMisgenesis},
  category={rank},
  description={locate ground for new settlements.
  To qualify, one must have sufficient skill at \glsentrytext{witchcraft} to wander past the \glsentrytext{edge} without fear},
}

\longnewglossaryentry{doulaShop}{
  name={Doula Shops},
  text={Doula Shop},
  sort={Doula Shop},
  prefix={a\space},
  category={location},
  sort={ZZ},
  parent={templeOfMisgenesis},
  symbol={\glssymbol{nulla}},
  description={sell various potions and blessings},
}

\longnewglossaryentry{paik}{
  name={Paik},
  category={god},
  symbol={$\Large\newmoon$},
  parent={ainumar},
  description={is the god of death by justice.
  When bandits swing from the noose in the \glsentrytext{court}, Paik takes them to his realm, and taunts them forever},
}

\longnewglossaryentry{templeOfJustice}{
  name={The Temple of Justice},
  text={Temple of Justice},
  plural={Temples of Justice},
  prefix={a\space},
  symbol={\glsentrysymbol{paik}},
  parent={temple},
  nonumberlist,
  description={Left to their own paranoia, people form mobs, and mob-justice prevails.
  This temple thwarts the worst plans of the god \glsentrytext{paik} by providing impartial, official, justice},
}

\longnewglossaryentry{court}{
  name={The Pit of Justice},
  text={Pit of Justice},
  prefix={a\space},
  category={location},
  sort={ZZ},
  parent={templeOfJustice},
  description={is where a town's \glsentrytext{warden} resolves legal disputes, and decides on the correct punishment for criminals.
  All trials must be on display, to warn the people about the consequences of crime, so they can learn that justice always prevails.
  And all trials must be entertaining, or nobody will pay the entry fee of 1~\glsentrytext{cp}},
}

\longnewglossaryentry{keeper}{
  name={Keepers},
  text={Keeper},
  nonumberlist,
  parent={templeOfJustice},
  category={profession},
  sort={3},
  prefix={a\space},
  description={bear the heavy burden of enforcing laws, and maintaining the \glsentrytext{court}.
  Every \glsentrytext{village} with a couple of hundred souls has a keeper to keep them right, and collect payments for the vital service they provide},
}

\longnewglossaryentry{warden}{
  name={Wardens},
  text={Warden},
  nonumberlist,
  parent={templeOfJustice},
  category={rank},
  prefix={a\space},
  description={make the laws and decide fitting punishments for criminals in their \glsentrytext{court}},
}

\longnewglossaryentry{seneschal}{
  name={Seneschals},
  text={seneschal},
  parent={templeOfJustice},
  category={rank},
  prefix={a\space},
  description={count every \glsfmtlongpl{cp} in their \glsentrytext{warden}'s, keep their \glsentrytext{warden} up-to-date on recent affairs, and keep in contact with every \glsentrytext{guard} \glsentrytext{jotter} in the area, and ensure the \glsentrytext{wheatGuild} don't cause problems},
}

\longnewglossaryentry{sunGuard}{
  name={Sun Guards},
  sort={Sun Guard},
  text={Sun Guard},
  prefix={a\space},
  parent={templeOfJustice},
  category={rank},
  description={are the upstanding soldiers who protect the city from all the unwashed masses, while wearing shiny-white tabards},
}

\longnewglossaryentry{sable}{
  name={Sable},
  category={god},
  symbol={$\large\hexstar$},
  parent={ainumar},
  description={releases cold into the world to watch people lay down and die in the snow, then takes them to his frigid realm, to place their frozen spirits there like an ornament},
}

\longnewglossaryentry{templeOfFrost}{
  name={The Temple of Frost},
  text={Temple of Frost},
  plural={Temples of Frost},
  prefix={a\space},
  symbol={\glsentrysymbol{sable}},
  parent={temple},
  nonumberlist,
  description={exists to thwart \glsentrytext{sable}, who comes to \glsentrytext{fenestra} every year to steal souls with frost.
  Before he strikes, the weavers provide preservatives.
  As snow begins to fall, everyone approaches the weavers for warm clothes, or just to sit by their warm fire and gossip for a while},
}

\longnewglossaryentry{weaversGuild}{
  name={The Weavers' Guild},
  text={Weavers' Guild},
  sort={Weavers' Guild},
  prefix={a\space},
  category={location},
  sort={ZZ},
  parent={templeOfFrost},
  description={houses a massive fire, many looms, and plenty of space to tell a long story},
}

\longnewglossaryentry{weaver}{
  name={Weavers},
  text={Weaver},
  prefix={a\space},
  parent={templeOfFrost},
  category={profession},
  sort={3},
  description={produce the best and warmest of clothes for \glsentrytext{cTwo}},
}

\longnewglossaryentry{notary}{
  name={Notaries},
  text={notary},
  plural={notaries},
  prefix={a\space},
  parent={templeOfFrost},
  category={rank},
  description={records transferrance of ownership, so people can save money, without having to transport valuable goods.
  The real items remain in a secret location, where they won't interfere with trade},
}

\longnewglossaryentry{wrecan}{
  name={Wrecan},
  category={god},
  symbol={\Amor},
  parent={ainumar},
  description={is the goddess of hatred, bigotry, and vengeance.
  When people fight each other, she takes their soul to her realm of eternal war},
}

\longnewglossaryentry{templeOfHate}{
  name={The Temple of Hate},
  text={Temple of Hate},
  plural={Temples of Hate},
  prefix={a\space},
  symbol={\glsentrysymbol{wrecan}},
  parent={temple},
  nonumberlist,
  description={As resources run thin, humanity becomes its own worst enemy.
  Disgust spreads into bigotry, then war, as \glsentrytext{wrecan} pulls people into her realm},
}

\longnewglossaryentry{armourHall}{
  name={The Armourers' Guild},
  text={Armourers' Guild},
  prefix={an\space},
  category={location},
  sort={ZZ},
  parent={templeOfHate},
  description={take in angry, young people, and redirect their anger into work and negotiation skills.
  They handle all the major disputes which nobody wants to take to the \glsentrytext{templeOfJustice}, and often become the de-facto arbiters when one \glsentrytext{warden} cannot agree with another},
}

\longnewglossaryentry{armourer}{
  name={Armourers},
  text={Armourer},
  prefix={an\space},
  parent={templeOfHate},
  category={profession},
  sort={3},
  nonumberlist,
  description={work in the \glsentrytext{armourHall} and guard civilization from death by Hate, with a combination of excellent armour, and diplomatic skills},
}

\longnewglossaryentry{proctor}{
  name={Proctors},
  text={proctor},
  prefix={a\space},
  parent={templeOfHate},
  category={rank},
  sort={3},
  nonumberlist,
  description={settle disputes between anyone who agrees to their arbitration.
  Simple missions involve a house-visit, more famed proctors journey the \glsentrytext{lonelyRoad} to ensure some distant \glsentrytext{warden} changes his behaviour.
  When all else fails, proctors have the abililty to ban the sale of armour across a town and nearby settlements.

  Most only serve for a year before finding someone else to take the job},
}

\longnewglossaryentry{yonder}{
  name={Yonder},
  category={god},
  symbol={$\Large\bell$},
  parent={ainumar},
  description={is the god who kills by curiosity.
  When idiots go to investigate something which sensible people would leave alone, they say their soul goes to live with \glsentrytext{yonder}.
  Nobody knows what happens after that, and there's only one way to find out\ldots},
}

\longnewglossaryentry{templeOfCuriosity}{
  name={The Temple of Curiosity},
  text={Temple of Curiosity},
  plural={Temples of Curiosity},
  prefix={a\space},
  symbol={\glsentrysymbol{yonder}},
  parent={temple},
  nonumberlist,
  description={People who go searching for answers often don't come back.
  This temple keeps an official log of all curiosities so that people don't have to go anywhere dangerous to learn -- they can just read.
  Anyone not content to stay inside and read eventually goes to write a travel-book},
}

\longnewglossaryentry{paperGuild}{
  name={The Paper Guild},
  text={Paper Guild},
  prefix={a\space},
  category={location},
  sort={ZZ},
  parent={templeOfCuriosity},
  description={makes maps, candles, paper, and soap.
  They always need more basilisk bodies to make the latter two},
}

\longnewglossaryentry{secretLibrary}{
  name={The Secret Library},
  text={the Secret Library},
  category={location},
  sort={ZZB},
  parent={templeOfCuriosity},
  description={controls the \glsentrytext{templeOfCuriosity} from an unknown location.
  It holds and hoards every map, secret, and oddity that any \glsentrytext{scribe} finds, and needs a constant supply of candles, so it can never go dark},
}

\longnewglossaryentry{scribe}{
  name={Scribes},
  text={Scribe},
  prefix={a\space},
  parent={templeOfCuriosity},
  category={profession},
  sort={3},
  nonumberlist,
  description={are anyone who works for the \glsentrytext{paperGuild}.
  Most spend their time making books or soap},
}

\longnewglossaryentry{chandler}{
  name={Chandlers},
  text={chandler},
  sort={1},
  prefix={a\space},
  parent={templeOfCuriosity},
  category={rank},
  description={make candles, in a hot, humid, basement.
  Everyone in the \glsentrytext{paperGuild} must begin here, in order to understand the effort light demands},
}

\longnewglossaryentry{seeker}{
  name={Seekers},
  text={Seeker},
  prefix={a\space},
  sort={6},
  parent={templeOfCuriosity},
  category={rank},
  description={travel to deliver messages, and gather information for the \glsentrytext{paperGuild}},
}

\longnewglossaryentry{cartographer}{
  name={Cartographers},
  text={cartographer},
  sort={5},
  prefix={a\space},
  parent={templeOfCuriosity},
  category={rank},
  description={create, confirm, and update maps.
    The most valuable of these contain information on \glsentrytext{ingredient}-rich ground, which they use for \glsentrytext{alchemy} spells to learn about the landscape, so they can make more maps},
}

\longnewglossaryentry{librarian}{
  name={Librarians},
  text={librarian},
  sort={6},
  prefix={a\space},
  parent={templeOfCuriosity},
  category={rank},
  description={run libraries.
  The previous librarian selects them},
}

\longnewglossaryentry{philosopher}{
  name={Philosophers},
  text={philosopher},
  sort={7},
  prefix={a\space},
  parent={templeOfCuriosity},
  category={rank},
  description={write books about what the previous Philosopher tried to say, in language so precise that nobody else can read them},
}

%%%%% Seasons

\longnewglossaryentry{cOne}{
  name={Niquis},
  prefix={a\space},
  sort={1},
  parent={cycle},
  description={is the first \glsentrytext{cycle}, and start with mild weather, and after three days an eclipse blots out the \glsentrytext{ainumar}.

  At the end, an unusually warm \glsentrytext{storm} allows any cold-blooded \glsentrytext{monster} one last opportunity to gorge before the temperature drops rapidly},
}

\longnewglossaryentry{cTwo}{
  name={Sables},
  prefix={a\space},
  sort={2},
  parent={cycle},
  description={begins the second \glsentrytext{cycle} with a warm \glsentrytext{storm}, then immediately after brings cold snow, then more each day.

  At the half-way point -- day thirty -- a fifteen-hour eclipse plumets \glsentrytext{fenestra} into the black, frozen abyss of space.
  People say the eclipse represent's \glsentrytext{sable}'s attempt to destroy \glsentrytext{fenestra}},
}

\longnewglossaryentry{cThree}{
  name={Halkin},
  prefix={a\space},
  sort={3},
  parent={cycle},
  description={begins the third \glsentrytext{cycle} with a long \glsentrytext{storm} as \glsentrytext{grummel} pulls \glsentrytext{fenestra} back from the shadows to consume more souls and awaken every \glsentrytext{basilisk} from hibernation.

  Three days before the end, an eclipse marks \glsentrytext{sable}'s retreat as the last \glsentrytext{basilisk} awakens},
}

\longnewglossaryentry{cFour}{
  name={Umba},
  prefix={an\space},
  sort={4},
  parent={cycle},
  description={works like Spring, bringing warmth, lambs, and myriad insects.
  As the fourth \glsentrytext{cycle} out of six, it marks the half-way point, when the weather becomes warmer},
}

\longnewglossaryentry{cFive}{
  name={Sylfs},
  prefix={a\space},
  sort={5},
  parent={cycle},
  description={begins the fifth \glsentrytext{cycle} with a cold-snap eclipse, just as the \glsentrytext{storm} hits, then quickly becomes scorching hot.
  No sane creature stays in the Sun for long during this \glsentrytext{cycle}},
}

\longnewglossaryentry{cSix}{
  name={Lantalka},
  prefix={a\space},
  sort={6},
  parent={cycle},
  description={begins the last \glsentrytext{cycle} with an eclipse and a sudden temperature drop.
  The next day returns to feeling warm, but never quite as warm as in \glsentrytext{cFive}.

  Over the next sixty days, the temperature slowly falls, until \glsentrytext{cOne} begins a new year},
}

% Weather

\longnewglossaryentry{weather}{
  name={Weather},
  prefix={a\space},
  parent={travel},
  type={mech},
  description={can spell death for anyone who doesn't take it seriously.
  Some of these cosmological terrors will lay waste to the underprepared, while others sneak up on them slowly},
}

\longnewglossaryentry{storm}{
  name={\Glsentrytext{grummel}'s Wrath},
  text={wrath},
  prefix={a\space},
  type={mech},
  parent={cosmology},
  description={brings the face of the \glsentrytext{ainumar} up close at the end of each \glsentrytext{cycle}.
  As the face of the gods looks a dozen-times larger than usual, the ground quakes, lightning strikes, the wind rips houses apart, and tidal waves thrash at every shore.
  During this time, underground people venture deeper underground, while land-dwelling people often leave their houses in search of an empty space.

  Any structure not built to withstand the quakes has a 1 in 6 chance of toppling, and travel speeds reduce to half.

  No sane creature lives by the sea in \glsentrytext{fenestra} -- the tidal waves remove all civilization from every shore},
}

\longnewglossaryentry{coldSnap}{
  name={Cold Snaps},
  text={cold snap},
  prefix={a\space},
  type={mech},
  parent={weather},
  description={can drain the life from characters and wildlife in a blink.
  During the cold, each \glsentrytext{interval} without protection inflicts 2~\glsfmtplural{ep}.

  The mass disturbances in the air, mixed with the darkness, creates a kind of magic in the air, which grants +1 to everyone's Fate \glsentrytext{sphere}},
}

\longnewglossaryentry{earthquake}{
  name={Earthquakes},
  text={earthquake},
  prefix={an\space},
  type={mech},
  parent={weather},
  description={can topple \glsentrytext{village} walls, houses collapse inward, and gnomish warrens have their planning tested from the foundations to the rooves.
  Despite the shaking, most structures remain standing.
  Architects in \glsentrytext{fenestra} design castles for quakes, and dwarvish settlements, down at the edge of the \glsentrytext{deep}, often feel nothing, as underground caverns don't shake during quakes as badly as the surface does.

  When quakes break bedrock, everyone with the tiniest understanding of magic finds themselves able to speak to stone, at least a little, as rocks and ice start to wake up.
  This time grants a +1 bonus to everyone's Earth \glsentrytext{sphere}, for the short duration of the quake},
}

\longnewglossaryentry{eclipse}{
  name={Eclipses},
  text={eclipse},
  prefix={an\space},
  type={mech},
  parent={weather},
  description={plunge \glsentrytext{fenestra} into sudden daytime darkness, as the \glsentrytext{ainumar} blocks the Sun.
  For this entire \glsentrytext{interval}, the winds becomes more pliable, and anyone with a voice to speak gains +1 to their Air \glsentrytext{sphere}},
}

\longnewglossaryentry{flood}{
  name={Floods},
  text={flood},
  prefix={a\space},
  type={mech},
  parent={weather},
  description={ damage infrastructure worse than earthquakes.
  They rot food, and degrade the foundations of houses in subtle ways, which only become apparent years later.
  High fortifications remain untouched, but underground dwelling runs the risk of water pouring in from above, and driving everything inside up into the Sunlight.

  \Glsentrytext{fenestra}'s predators seem to have an instinct for floods, and will camp outside any homely holes in their territories.

  Travel during a flood poses serious problems -- any affected area will half the travellers' rate of movement.

While floods occur, everyone gains a +1 Bonus to their Water \glsentrytext{sphere}},
}

\longnewglossaryentry{heatwave}{
  name={Heatwave},
  text={heatwave},
  prefix={a\space},
  type={mech},
  parent={weather},
  description={will turn most armour into a liability, as all travellers in direct Sunlight will endure 1~\glsentrytext{ep} each \glsentrytext{interval} while wearing loose-fitting clothes, or +2 in heavy clothes.

  During a heatwave, everyone gains a +1 Bonus to their Fire \glsentrytext{sphere}},
}

\longnewglossaryentry{hurricane}{
  name={Hurricanes},
  text={hurricane},
  prefix={a\space},
  type={mech},
  parent={weather},
  description={make travel challenging, and reduce the standard miles travelled by 1.
  All large items -- rooves, carts, et c. -- have a 1 in 6 chance of getting pulled off and possibly thrown by the wind},
}

\longnewglossaryentry{landslide}{
  name={Landslides},
  text={landslide},
  prefix={a\space},
  type={mech},
  parent={weather},
  description={will block all travel across an area for one full \glsentrytext{interval}, and everyone in the area has to flee or seek shelter.
  Groups use a single \glsentrytext{natural} roll of \roll{Speed}{Athletics} (\tn[10]) to determine the outcome for everyone.
  Anyone who fails becomes buried; they break free with \roll{Strength}{Survival} (\tn[10]), or remain buried forever},
}

\longnewglossaryentry{snow}{
  name={Snow},
  text={snow},
  plural={snow},
  prefix={a\space},
  type={mech},
  parent={weather},
  description={slows every \glsentrytext{journey} to half the standard speed.
  The added cold also inflicts 1~\glsentrytext{ep} each \glsentrytext{interval} a character spends exposed},
}

%%%%% Plants

\longnewglossaryentry{plant}{
  name={Plants},
  text={plant},
  prefix={a\space},
  symbol={\adfhangingflatleafleft},
  parent={fenestra},
  description={within \glsentrytext{fenestra} can grow large, strange, and deadly},
}

\longnewglossaryentry{bedshroom}{
  name={Bedshrooms},
  text={bedshroom},
  prefix={a\space},
  category={indexed},
  parent={plant},
  description={are fungi that look like shrivelled up cloth-sacks.
  While they have water, they grow spores inside the sack.
  Once they dry up, any disturbance releases large amounts of spores.
  Anyone breathing the spores in feels a dream rushing over them, beckoning them to sleep.
  They receive $1D3$~\glsfmtplural{ep} and roll \roll{Strength}{Academics} (\glsentrytext{tn} 8) to stay awake.
  Anyone who sleeps for the \glsentrytext{interval} removes $1D6$ \glsfmtplural{ep} and gains 1~\glsentrytext{mp}},
}

\longnewglossaryentry{dryadsKiss}{
  name={Dryad's Kiss Shrooms},
  text={dryad's kiss},
  plural={dryad's kisses},
  prefix={a\space},
  category={indexed},
  parent={plant},
  description={are a tasty mushroom which makes people very gullible.
    Anyone eating one takes a -2 penalty to all Deceit and Empathy checks for a day},
}

\longnewglossaryentry{glowshroom}{
  name={Glowshrooms},
  text={glowshroom},
  prefix={a\space},
  category={indexed},
  parent={plant},
  description={are subterranean fungi that give off a soft, faint light, but only in complete darkness.
    Dwarves sometimes use them instead of torches, even though the light is dimmer.

    Ingesting these plants can be deadly.
    While healthy to eat, after two \glsfmtplural{interval} they mix with stomach acids and begin to glow.
    This can turn someone into a sudden target in the dark, as their stomach shines faintly},
}

\longnewglossaryentry{marchingMushroom}{
  name={Marching Mushrooms},
  text={marching mushroom},
  plural={marching shrooms},
  prefix={a\space},
  category={indexed},
  parent={plant},
  description={relieve tiredness when chewed, but slow the body and mind.
    Once eaten, people ignore \glsentrytext{ep} Penalties for $1D6$~\glsfmtplural{interval}, but with -1 Penalty to Dexterity, Speed, Intelligence and Wits},
}

\longnewglossaryentry{horseDrops}{
  name={Horse-Drops},
  text={horse-drop moss},
  plural={horse-drops},
  prefix={some\space},
  category={indexed},
  parent={plant},
  description={are red little patches of moss, which grow on the roadside.
  Horses, donkeys, and mules love them, because they also love being eaten by those fast-moving animals.
  Once something eats the moss, it begins to grow inside the beast's stomach rapidly (inflicting one \glsentrytext{ep} each \glsentrytext{interval}).
  Curing the beast requires an \roll{Intelligence}{Cultivation} roll (\tn[12]).
  Once the beast dies, it explodes in a shower of red spores, and anyone breathing enough in can suffer the same effects},
}

\longnewglossaryentry{disgnome}{
  name={Disgnome Blooms},
  text={disgnome bloom},
  prefix={a\space},
  category={indexed},
  parent={plant},
  description={have yellow petals the size of a forearm, which sprout only in \glsentrytext{cFive}.
  The rest of the time, they look like any other plain bush.
  The roots sprout vicious, curved, spikes, with a mind-bending toxin.
  A single scratch inflicts $1D3$~\glsfmtplural{ep}, each of which inflict a -1 Wits Penalty for as long as that \glsentrytext{ep} remains},
}

%%%%% Symbols


\newglossaryentry{encumbrance}{
  type=symbols,
  sort=AE,
  nonumberlist,
  name={\Semisextile},
  description={Encumbrance},
}

\newglossaryentry{N}{
  type=symbols,
  sort=RZ,
  nonumberlist,
  name={\Hygiea},
  description={Goblinoid},
}

\newglossaryentry{R}{
  type=symbols,
  sort=TM,
  nonumberlist,
  name={\NorthNode},
  description={Morphed creature},
}

\newglossaryentry{D}{
  type=symbols,
  sort=TU,
  nonumberlist,
  name={\Lilith},
  description={Undead creature},
}

\newglossaryentry{T}{
  type=symbols,
  sort=TT,
  nonumberlist,
  name={\Opposition},
  description={A team of multiple creatures},
}

\newglossaryentry{E}{
  type=symbols,
  sort=SN,
  nonumberlist,
  name={\Mercury},
  description={Sentient (any gender or none)},
}

\newglossaryentry{F}{
  type=symbols,
  sort=SF,
  nonumberlist,
  name={\Venus},
  description={Female},
}

\newglossaryentry{M}{
  type=symbols,
  sort=SM,
  nonumberlist,
  name={\Mars},
  description={Male},
}

\newglossaryentry{A}{
  type=symbols,
  sort=TA,
  nonumberlist,
  name={\Taurus},
  description={Animal},
}

% RACES

\newglossaryentry{Dw}{
  type=symbols,
  sort=RDwarf,
  nonumberlist,
  name={\Vulkanus},
  description={Dwarf},
}

\newglossaryentry{El}{
  type=symbols,
  sort=RElf,
  nonumberlist,
  name={\Moon},
  description={Elf},
}

\newglossaryentry{Hu}{
  type=symbols,
  sort=RHuman,
  nonumberlist,
  name={\Saturn},
  description={Human},
}

\newglossaryentry{Gn}{
  type=symbols,
  sort=RGnome,
  nonumberlist,
  name={\Kronos},
  description={Gnome},
}

\newglossaryentry{Nl}{
  type=symbols,
  sort=RGnoll,
  nonumberlist,
  name={\Admetos},
  description={Gnoll},
}

\newglossaryentry{morning}{
  type=symbols,
  sort=T1,
  nonumberlist,
  name={\Leftcircle},
  description={is first light, first \glsentrytext{interval}},
}

\newglossaryentry{afternoon}{
  type=symbols,
  sort=T2,
  nonumberlist,
  name={\Circle},
  description={is the second \glsentrytext{interval}},
}

\newglossaryentry{evening}{
  type=symbols,
  sort=T3,
  nonumberlist,
  name={\RIGHTCIRCLE},
  description={begins darkness, in the day's third \glsentrytext{interval}},
}

\newglossaryentry{night}{
  type=symbols,
  sort=T4,
  nonumberlist,
  name={\CIRCLE},
  description={means darkness over the fourth \glsentrytext{interval}},
}

\newglossaryentry{squash}{
  type=symbols,
  sort=ASquash,
  nonumberlist,
  name={\Vesta},
  description={Play this \glsentrytext{sq} at the same time as the available one (i.e. not from the same \glsentrytext{sq})},
}

\newglossaryentry{sqn}{
  type=symbols,
  sort=Asqn,
  nonumberlist,
  name={\Square},
  description={Side Quest is not ready yet},
}

\newglossaryentry{sqr}{
  type=symbols,
  sort=Asqr,
  nonumberlist,
  name={\CheckedBox},
  description={Side Quest is ready},
}

\newglossaryentry{vlg}{
  type=symbols,
  name={\Psyche},
  description={\glsentrytext{segment} removes something from the map}
}

% SHORT COMMANDS

\newcommand{\T}[1][1]{\gls{T}\setcounter{noAppearing}{#1}} % groups
\newcommand{\M}{\gls{M}} % male
\newcommand{\F}{\gls{F}} % female
\newcommand{\E}{\gls{E}} % sentient
\newcommand{\A}{\gls{A}} % creature
\newcommand{\Pl}{\glssymbol{plant}} % plant
\newcommand{\Sw}{\glssymbol{swarm}} % swarm
\newcommand{\N}{\gls{N}} % nura
\newcommand{\R}{\gls{R}} % morph
\newcommand{\D}{\gls{D}} % undead
\newcommand{\Dw}{\gls{Dw}} % dwarf
\newcommand{\El}{\gls{El}} % elf
\newcommand{\Hu}{\gls{Hu}} % human
\newcommand{\Gn}{\gls{Gn}} % gnome
\newcommand{\Nl}{\gls{Nl}} % gnoll
\newcommand{\squash}{\gls{squash}} % multi-part side quest
\newcommand{\sqr}{\glsentrytext{sqr}} % multi-part side quest
\newcommand{\sqn}{\glsentrytext{sqn}} % multi-part side quest

