\ifdefined\glossarytitle
  \relax
\else
  \newcommand\glossarytitle{Lexicon}
\fi

% Alter topicmcols style

%\renewcommand*{\glstopicPostSkip}{---}
\renewcommand*{\glstopicPreSkip}{\bigLine\medskip}
\renewcommand*{\glstopicSubPreLocSep}{~\adforn{38}~}
\renewcommand*{\glstopicInit}{\setlength{\parskip}{10pt}}



\newglossary*{people}{People}
\glssetcategoryattribute{people}{indexonlyfirst}{true}

\newglossary*{mech}{Mechanics}
\glssetcategoryattribute{mech}{indexonlyfirst}{true}

\renewcommand*{\acronymtype}{mech}

\glsdefpostdesc{general}{.}
\glssetcategoryattribute{general}{glossnamefont}{textbf}
\glsdefpostdesc{people}{.\space\adforn{15}\space}
\glsdefpostdesc{rules}{.\space\adforn{35}\space}
\glsdefpostdesc{god}{.}
\glsdefpostdesc{acronym}{.}
\glsdefpostdesc{location}{.\quad\adforn{14}\space}
\glsdefpostdesc{symbol}{.}

\glssetcategoryattribute{god}{glossnamefont}{textsf}

\newignoredglossary*{peeps}

\glssetcategoryattribute{rules}{nonumberlist}{true}
\glssetcategoryattribute{rules}{noindex}{true}
\glssetcategoryattribute{rules}{glossnamefont}{textsc}
\glssetcategoryattribute{rules}{dualindex}{true}
\glssetcategoryattribute{abbreviation}{glossnamefont}{textit}

\glssetcategoryattribute{location}{glossnamefont}{emph}

\makeglossaries

% We only want to show full descriptions (e.g. 'PC (Player Character)') in some books. These books make the file '.switch-gls', but most use nothing.

\IfFileExists{.switch-gls}{
  \setabbreviationstyle[acronym]{long-short-sc-desc}
}{
  \setabbreviationstyle[acronym]{short-sc-desc}
}


%%%%%%%%%%

%%%%%%%%%% Meta Terminology %%%%%%%%%%

\newacronym[
  description={a brand GNU RPG},
  sort={T1},
  name={BIND},
  nonumberlist,
]{bind}{BIND}{BIND Is Not DnD}

\newacronym[
  description={books work a little like `print on demand', but it's faster, and cheaper, and if you don't like the paper-quality, you have only yourself to blame},
  prefix={a\space},
  parent={bind},
  ]{piy}{PiY}{Print it Yourself}

\newacronym[
  description={means that the file is for printing, not for reading on a screen},
  prefix={a\space},
  parent={bind},
  nonumberlist,
  ]{pdf}{pdf}{Printable Document Format}

% Traits

\longnewglossaryentry{trait}{
  sort={T2},
  name={Traits},
  text={Trait},
  prefix={a\space},
  type={mech},
  description={are numbers and Knacks},
  }

\longnewglossaryentry{attribute}{
  name={Attributes},
  text={Attribute},
  prefix={an\space},
  type={mech},
  parent={trait},
  first={\textit{Attribute}},
  description={describe the body and mind.
    \par
    \vspace{1em}
    \noindent
    \begin{minipage}{\linewidth}
    \begin{description}
      \item[Strength:]
      muscle, brawn, toughness, height
      \item[Dexterity:]
      finesse, coördination, balance
      \item[Speed:]
      velocity, tendons, vim
      \item[Intelligence:]
      memory, logic, tenacity, cunning
      \item[Wits:]
      alacrity, levity, attention, acumen
      \item[Charisma:]
      gravitas, glamour, confidence, symmetry
    \end{description}
    \end{minipage}
    Players can remove penalties with minimal \glsentrytext{xp} expenditure, but the price hike after that grows steeply},
}

\longnewglossaryentry{skill}{
  name={Skills},
  text={Skill},
  prefix={a\space},
  type={mech},
  parent={trait},
  description={show training.  Characters perform tasks by rolling $2D6$ plus \glsentrytext{attribute} plus Skill.
  Each Skill helps with many tasks, depending on the \glsentrytext{attribute} paired with},
  }


% Time

\longnewglossaryentry{campaign}{
  sort={T10},
  name={Chronicles},
  text={Chronicle},
  prefix={a\space},
  type={mech},
  description={are series of games, linked together by \glsentrytext{sq} plots, and every players' \glsentrytext{characterPool}},
  }

\newacronym[
  description={-- one of the characters run by the people playing the game},
  parent={campaign},
  prefix={a\space},
  ]{pc}{PC}{Player Character}

\longnewglossaryentry{gm}{
  prefix={a\space},
  name={The Judge},
  text={Judge},
  sort={Judge},
  type={mech},
  parent={campaign},
  nonumberlist,
  description={rolls encounters, interprets the rules, and forgets to bring enough pencils},
}

\newacronym[
  description={ -- anyone in the world played by the \glsentrytext{gm} rather than a player},
  prefix={an\space},
  prefixfirst={a\space},
  parent={campaign},
  ]{npc}{NPC}{Non-Player Character}

\longnewglossaryentry{downtime}{
  name={Downtime},
  text={Downtime},
  prefix={a\space},
  parent={campaign},
  type={mech},
  description={between sessions lets characters train, heal, and drink.
  Every week in our world, three weeks pass in \glsentrytext{fenestra}.

  Characters recover half of their current \glspl{hp} (minimum 1), each week},
  }

\longnewglossaryentry{sq}{
  name={Side Quests},
  text={Side~Quest},
  prefix={a\space},
  parent={campaign},
  type={mech},
  description={are \glsentrytext{bind}'s way of structuring stories.
  Each \gls{segment} has a location (e.g. `Town', or `Forest'), where it `activates' once the troupe enter the area},
  }

\longnewglossaryentry{segment}{
  name={Segments},
  text={segment},
  prefix={a\space},
  parent={sq},
  type={mech},
  description={are scenes which form a larger story},
  }

\longnewglossaryentry{interval}{
  name={Intervals},
  text={Interval},
  prefix={an\space},
  parent={campaign},
  sort={Interval},
  type={mech},
  nonumberlist,
  description={divide the day into four parts -- morning (\showInterval{0}), afternoon (\showInterval{1}), evening (\showInterval{2}), and night (\showInterval{3}).
    \par
    \vspace{1em}
    \noindent
    \begin{description}
      \item[Resting characters] remove \pgls{ep}.
      \item[The \gls{gm}] rolls $1D6$ -- everyone gains that many \glspl{fp}.
      \item[The wind] brings \glspl{mp} towards the biggest vacuum.
    \end{description}
    \manaRegenChart
    Each day, the troupe must eat and sleep, or carry an extra~\glsentrytext{ep}},
}

% Stories

\longnewglossaryentry{storypoint}{
  sort={T13},
  name={Story Points},
  text={Story Point},
  type={mech},
  prefix={a\space},
  description={allow players to declare that some part of their backstory arrives on scene to help the situation. They also grant \glsfmtlongpl{xp}},
}

\newacronym[
  description={come from each character's Code.
  Spend them to raise any \glsentrytext{trait}.
    \noindent
    \begin{boxtable}
         \textbf{Trait} & \textbf{Cost} \\
         \hline
         \glsentrytext{attribute} & $5^{n} +5$  \\
         \glsentrytext{skill}     & $5\times n$ \\
         Knack                    & $5\times n$ \\
         Martial \glsentrytext{skill} & $10\times n$ \\
    \end{boxtable}
  Here `$n$' is the level purchased},
  shortplural={XP},
  prefix={an\space},
  name={Experience Points (XP)},
  parent={storypoint},
  nonumberlist,
  ]{xp}{XP}{Experience Point}

\longnewglossaryentry{characterPool}{
  name={Character Pool},
  prefix={a\space},
  type={mech},
  category={rules},
  parent={storypoint},
  description={is the collection of characters a player has.
  Players introduce new characters as allies by spending \glsfmtplural{storypoint}, then once their character dies, they take their next \glsentrytext{pc} from the pool},
}

% Space

\longnewglossaryentry{region}{
  sort={T4},
  name={Regions},
  text={Region},
  prefix={a\space},
  type={mech},
  category={rules},
  nonumberlist,
  description={are broad areas like `forest', `town', `roads'.
  Each contains different encounter types, and often has its own \glsentrytext{sq} collection},
  }

\longnewglossaryentry{area}{
  name={Areas},
  sort={Area},
  text={Area},
  prefix={an\space},
  parent={region},
  type={mech},
  category={rules},
  nonumberlist,
  description={give a rough unit for large spaces.
  An area is a space made distinct by its features.
  In the \glsentrytext{deep}, each cavern might count as an area, while out in the open plains a forest might be composed of the local areas: `the centre with the big, felled tree', `the river's fork', and `the griffins' nesting site'},
  }


\longnewglossaryentry{step}{
  name={Steps},
  text={step},
  prefix={a\space},
  parent={region},
  type={mech},
  category={rules},
  nonumberlist,
  description={provide a rough measure of space.
  We can imagine it about a metre long, or as wide as the step on your gaming board, or any other length},
  }

% Actions

\longnewglossaryentry{action}{
  sort={T5},
  name={Actions},
  text={Action},
  type={mech},
  category={rules},
  prefix={an\space},
  description={for dangerous prizes resolve with $2D6$ plus \glsentrytext{attribute}, plus \glsentrytext{skill}.
    If the roll beats the \glsentrytext{tn}, the character gets the prize;
    if the roll is under, they get the danger.
    When a roll hits the \glsentrytext{tn}, the player chooses both or neither},
}

\newacronym[
  description={means the number players need to roll on the dice to achieve a \emph{tie} with the task.
    Rolling higher indicates they have their prize, rolling lower means some nasty outcome is upon them,
    \noindent
    \begin{boxtable}
         \textbf{TN} & \textbf{Difficulty} \\
         \hline
         6 & Easy -- just ask the barmaid what you want. \\
         7 & Basic -- find firewood in the forest. \\
         10 & Tricky -- find a good price in the market. \\
         12 & Professional -- fix the cart by Sundown. \\
         14 & Specialist -- Plan a three-storey stone building. \\
    \end{boxtable}
  and rolling a tie means both (or neither)},
  shortplural={TNs},
  sort={S1},
  parent={action},
  prefix={a\space},
  ]{tn}{TN}{Tie Number}

\longnewglossaryentry{natural}{
  name={Natural Rolls},
  text={Natural Roll},
  sort={Natural Roll},
  prefix={a\space},
  sort={S2},
  parent={action},
  type={mech},
  category={rules},
  nonumberlist,
  description={represent the situation, and stay where they are; later rolls need to use the same result.

  If someone tries to figure out how to find their way out of the forest, and back to a road, the player could roll `\twoDice{4}'.
  With a +1 Bonus, the total is `5'.
  The next character has a +3 Bonus, so their total is `7'.
  With the \glsentrytext{tn} set at `10', the group cannot find their way back without changing their approach},
  }

\longnewglossaryentry{resistedaction}{
  name={Resisted Actions},
  text={Resisted Action},
  sort={S3},
  parent={tn},
  type={mech},
  category={rules},
  prefix={a\space},
  description={start at \glsentrytext{tn}~7, then add the \glsentrytext{npc}'s Bonuses.
    For example, a player declares their \glsentrytext{pc} wants to demand a new sword, but the \glsentrytext{gm} thinks the \glsentrytext{jotter} will just reflexively lie about supplies running low.

    The \glsentrytext{jotter}'s \roll{Wits}{Deceit} come to~+2 in total, so the \glsentrytext{tn} is ($7 + 2 =$) 10},
}

\longnewglossaryentry{restingaction}{
  name={Resting Actions},
  text={Resting Action},
  sort={Resting Action},
  sort={S6},
  parent={action},
  type={mech},
  category={rules},
  prefix={a\space},
  description={apply when you can repeat something, without danger.
Set the darker die to `\dicef{6}' and roll the other.  If this roll fails, it fails forever},
}

\longnewglossaryentry{bandAct}{
  name={Banding Actions},
  sort={Banding Action},
  text={Banding Action},
  prefix={a\space},
  sort={S6},
  parent={action},
  type={mech},
  category={rules},
  nonumberlist,
  description={means characters will get one better working together.
  In this case, the first character adds their Bonus, the second adds half, the third, a quarter, et c. and we round halves up at the end},
  }

% Combat

\longnewglossaryentry{combat}{
  sort={T6},
  name={Combat},
  text={Combat},
  type={mech},
  category={rules},
  prefix={a\space},
  description={uses standard resisted rolls.
  The attacker rolls \roll{Dexterity}{Martial Skill}, and the \glsentrytext{tn} equals 7 + the \glsentrytext{npc}'s \roll{Dexterity}{Martial Skill}},
}

\longnewglossaryentry{quickaction}{
  name={A Response Action},
  text={Response Action},
  prefix={a\space},
  sort={S3},
  parent={combat},
  type={mech},
  category={rules},
  nonumberlist,
  description={means the character must resist some \glsentrytext{resistedaction}.
  If the \glsentrytext{ap} loss push them below 0, then every negative becomes a penalty to all action},
}

\newacronym[
  description={give a rough estimate of a creature's combat abilities},
  shortplural={CRs},
  prefix={a\space},
  name={Creature Ratings (CR)},
  nonumberlist,
  parent={combat},
  ]{cr}{CR}{Combat Rating}

\newacronym[
  description={provide linear, measure of a character's health and injury},
  shortplural={HP},
  prefix={an\space},
  prefixfirst={a\space},
  parent={combat},
  name={Health Points (HP)},
  ]{hp}{HP}{Health Point}

\longnewglossaryentry{round}{
  name={Rounds},
  text={round},
  prefix={a\space},
  parent={combat},
  category={rules},
  type={mech},
  category={rules},
  description={start when everyone wants to speak at once.
  The \glsentrytext{gm} goes round the table clockwise as players commit to actions by spending \glsentrytext{ap}-coins},
  }

\newacronym[
  description={measure how much luck the character has left, used solely to avoid Damage.
  $\Glspl{fp} = \frac{Total~\glsfmtplural{xp}}{10} + Charisma$},
  shortplural={FP},
  prefix={an\space},
  parent={combat},
  name={Fate Points (FP)},
  ]{fp}{FP}{Fate Point}

\longnewglossaryentry{weapon}{
  name={Weapons},
  text={weapon},
  type={mech},
  prefix={a\space},
  nonumberlist,
  parent={combat},
  description={add to Attack and Damage.
  Smaller weapons only cost 1~\glsentrytext{ap} to use, while larger weapons cost more, but also have bigger Bonuses},
}

\longnewglossaryentry{armour}{
  name={Armour},
  text={armour},
  type={mech},
  prefix={an\space},
  nonumberlist,
  parent={combat},
  description={protects characters by reducing Damage.
  It takes effect before \glsentrytext{dr} applies},
}

\newacronym[
  description={represent armour of any type, or other states which help avoid Damage},
  shortplural={DR},
  parent={armour},
  prefix={a\space},
  ]{dr}{DR}{Damage Resistance}

\longnewglossaryentry{covering}{
  name={Covering},
  type={mech},
  category={rules},
  prefix={a\space},
  nonumberlist,
  parent={armour},
  description={means how much \glsentrytext{armour} covers the body.
  \Glsentrytext{armour} with `Covering 3' protects the torso and may have a helmet, while armour with `Covering 5' protects almost the entire body},
}

\longnewglossaryentry{vitalShot}{
  name={Vitals Shots},
  text={Vitals Shot},
  sort={Vitals Shot},
  prefix={a\space},
  nonumberlist,
  type={mech},
  category={rules},
  parent={armour},
  description={are attacks which equal a target's \glsentrytext{tn} plus their \glsentrytext{armour}'s \glsentrytext{covering}; this lets the attack ignore the \glsentrytext{armour}'s \glsentrytext{dr}, and deal direct Damage.

  If a player needs to roll at \glsentrytext{tn}~10 to hit an opponent with `\glsentrytext{covering}~3', then they need to roll `13' to make a Vitals Shot.
  This applies symmetrically; if the \glsentrytext{pc}'s armour has `\glsentrytext{covering}~5', and they miss by 5, then their opponent scores a Vitals Shot, and their armour counts for nothing, and does not provide any \glsentrytext{dr}},
  }

\longnewglossaryentry{swarm}{
  name={Swarms},
  text={swarm},
  type={mech},
  category={rules},
  symbol={\Juno},
  prefix={a\space},
  nonumberlist,
  parent={combat},
  description={
    are myriad tiny creatures, acting as one.
    They crawl over characters, and into gaps in armour.

    Swarms can cover a number of \glspl{step} equal to their \glsentrytext{hp}-total, or bunch up together, with 3~\glsentrytext{hp} per \gls{step}.

    Attacking swarms is easy when there are so many targets.
    The \glsentrytext{tn} to attack always reduces by 1 per \glsentrytext{hp} in the swarm, so when a swarm is listed with `{\scshape Att 12 - 8 \glsentrytext{hp}}', the \glsentrytext{tn} would be only 4; but if the swarm had only 1~\glsentrytext{hp} left, hitting it would require a roll at \glsentrytext{tn}~11.
    However, swarms only take 1 Damage each per attack.

    Swarms can split into smaller parts as a normal movement action.
    Each part inflicts 1~Damage each \glsentrytext{round} to anyone on the same \gls{step}, as long as the swarm's \glsentrytext{hp} total comes to more than the target's \glsentrytext{covering}},
}

\longnewglossaryentry{projectiles}{
  name={Projectiles},
  text={Projectiles},
  plural={Projectiles},
  type={mech},
  category={rules},
  prefix={a\space},
  parent={combat},
  description={rolls use \roll{Dexterity}{Projectiles}, and targets resist with \roll{Speed}{Vigilance}.
    Every 5 \glsfmtplural{step}' distance adds +1 to the \glsentrytext{tn}.
    When \pgls{pc} hits the \gls{tn} precisely, they miss their first target, but hit another target behind (if any)},
}

\longnewglossaryentry{crossbow}{
  name={Crossbows},
  text={crossbow},
  type={mech},
  prefix={a\space},
  parent={projectiles},
  description={only need 1~\glsentrytext{ap} to fire.
  They grant a Bonus to hit equal to 4 -\glsentrytext{weight} and deal Damage equal to their \glsentrytext{weight} -2, doubled.

  Reloading a crossbow requires 5~rounds, plus the weapon's Damage, and the user must have a Strength Bonus at least as high as the weapon's \glsentrytext{weight}},
}

\longnewglossaryentry{bow}{
  name={Hunting Bows},
  text={hunting bow},
  type={mech},
  prefix={a\space},
  parent={projectiles},
  description={deal any amount of Damage, depending on the bow, but cannot be pulled back by someone with a Strength Bonus lower than the Damage.
  The \glsentrytext{ap}~cost to pull one back equals 2 plus its Damage.

  The hunting bow gives a Bonus to hit equal to its Damage, if the archer has time to draw properly (i.e. they still have at least 1~\glsentrytext{ap} after firing).
  Flustered archers, take the weapon's Bonus as a penalty if they would not be able to fire in time},
}

\longnewglossaryentry{impromptuThrownWeapons}{
  name={Impromptu Thrown Weapons},
  text={impromptu thrown weapon},
  type={mech},
  prefix={an\space},
  parent={projectiles},
  description={receive a -2 penalty to hit and Damage, and a further -1~Penalty per \glsentrytext{step}~thrown},
}

% Weight

\longnewglossaryentry{weight}{
  sort={T8},
  name={Weight Rating},
  text={Weight},
  first={Weight Rating},
  prefix={a\space},
  type={mech},
  category={rules},
  nonumberlist,
  description={shows how easy something is to carry.
  Characters can carry items with a total Weight Rating equal to their \glsentrytext{hp} total.
  Creatures have a \glsentrytext{weight} equal to their own \glsentrytext{hp}},
}

\newacronym[
  description={measure how tired, hungry, and fed-up characters feel.
  Characters can put up with a number of Exhaustion Points equal to their \glsentrytext{hp}, after which they receive penalties to act},
  shortplural={EP},
  prefix={an\space},
  parent={weight},
  name={Exhaustion Points (EP)},
  ]{ep}{EP}{Exhaustion Point}

% Magic

\longnewglossaryentry{casting}{
  sort={T7},
  name={Castings},
  text={casting},
  prefix={a\space},
  parent={action},
  type={mech},
  category={rules},
  description={start by spending one \glsentrytext{mp} per spell level.
  The \glsentrytext{witch} then commands the target \glsentrytext{sphere}, rolling \roll{Charisma}{} the lowest \glsentrytext{skill} required.

  When `overpsending' on the \glsentrytext{invocation}, the \glsentrytext{witch} gains 1~\glsentrytext{ep} for each \glsentrytext{mp} they lack.

  If nobody resists a spell, the \glsentrytext{tn} usually depends on its target.
  Earth spells can affect ice far more easily than rocks, and Air spells can whip up a gale  easier when outdoors.

  If a caster can think of a way to use a spell to stop a sword, they can enter combat as usual, rolling at \glsentrytext{tn} 7 plus the \glsentrytext{npc}'s \roll{Dexterity}{Combat}.
  A battle-ready \glsentrytext{witch} might encourage a warrior's torch to burn his own face off, or make him forget what he wanted to do a moment before his sword comes down},
}

\newacronym[
  name={Mana Points (MP)},
  shortplural={MP},
  prefix={an\space},
  prefixfirst={a\space},
  parent={casting},
  description={work as the ``battery power'' of a magic user, which allows them to power spells},
  ]{mp}{MP}{Mana Point}

\longnewglossaryentry{spell}{
  name={Spells},
  text={spell},
  prefix={a\space},
  parent={casting},
  type={mech},
  category={rules},
  description={have a mind of their own.
  Once cast, they endure until they burn through themselves, or something destroys them.
  To stop a Fire spell, someone must put the fire out, and if an angry \glsentrytext{witch} makes antlers grow on someone's head, the only way to `dispel' them is with a boning knife.

  Long-range spells cannot be reigned in; if the range is
  \setcounter{spellCost}{4}\setRange%
  \spellRange, the spell will find the nearest target at that distance.

  Casters only select a spell's first target.
  The spell forks through the others like lightning, and may `arc' across any distances up to its original range.
  Water spells which hit a river will spread through the river, but a curse with an `area' of 4 will have to jump until it has found four people to inflict itself on.

  To learn a spell, the caster's \glsentrytext{skill} must match each \glsentrytext{sphere}},
}

%%%%%%%%%% Acronyms

\newacronym[
  description={are the smallest unit of currency},
  shortplural={cp},
  name={Copper Pieces ({\scshape cp})},
  type={main},
  longplural={copper pieces},
  prefix={a\space},
]{cp}{cp}{copper piece}

\newacronym[
  description={gets you 100 copper pieces},
  name={Silver Pieces ({\scshape sp})},
  type={main},
  sort={silver piece},
  longplural={silver pieces},
  shortplural={sp},
  prefix={a\space},
  ]{sp}{sp}{silver piece}

\newacronym[
  description={convert to ten silver, or a thousand copper pieces},
  shortplural={gp},
  longplural={gold pieces},
  name={Gold Pieces ({\scshape gp})},
  type={main},
  sort={Gold Piece},
  prefix={a\space},
]{gp}{gp}{Gold Piece}

\newacronym[
  description={measure how many actions someone can take in a round, based on how fast they can move and react.
  Start with 3 AP, plus your speed; put that many coins on your character sheet, and spend them each time you take an action},
  shortplural={AP},
  name={Action Points (AP)},
  parent={combat},
  prefix={an\space},
  ]{ap}{AP}{Action Point}

\newacronym[
  description={show magical shielding from the Force sphere},
  shortplural={SP},
  prefix={a\space},
  ]{SP}{SP}{Shield Point}

%%%%%%%%%%%%%%%% General Terms %%%%%%%%%%%%%%%%%%%%

\longnewglossaryentry{ainumar}{
  name={The Ainumar},
  text={Ainumar},
  plural={Ainumari},
  sort={Ainumar},
  nonumberlist,
  description={is a great orb in the sky, commonly supposed to be where the gods live.
  \glsentrytext{fenestra} circles around the Ainumar, as it circles the Sun.
  On clear days, the great `eye of the gods' becomes visible on its face},
}

\longnewglossaryentry{fenestra}{
  name={Fenestra},
  nonumberlist,
  description={This land, where elves, gnolls, and humans look up at trees, like ants moving through blades of grass.
  Predators larger than a horse hunt deer and people in the same way, so everyone travels together, and well-armed},
}

\newacronym[
  description={is the universal way to measure time in \glsentrytext{fenestra}, where one \glsentrytext{cycle} equals three `years'},
  name={Gnomish Machine Time (GMT)},
  prefix={a\space},
  parent={fenestra},
  type={main},
]{gmt}{GMT}{Gnomish Machine Time}

\longnewglossaryentry{alchemy}{
  prefix={an\space},
  name={Alchemy},
  symbol={\glsentrytext{R}},
  text={alchemy},
  sort={Alchemy},
  description={is what alchemists do},
}

\longnewglossaryentry{sphere}{
  name={Sphere},
  prefix={a\space},
  parent={alchemy},
  description={divide the world into meaningful parts.
  The five elemental Spheres are Fire, Air, Fate, Water, and Earth.
  Each pair joins its neighbours, making the high Spheres; Light, Death, Mind, Life, and Force},
}

\longnewglossaryentry{boon}{
  name={Concoctions},
  text={Concoction},
  prefix={a\space},
  sort={concoction},
  parent={alchemy},
  description={are liquids or powders which, when thrown in the air, hyper-charge the use of a single magic \glsentrytext{sphere}, for anyone present next to the burst.
  For example, a concoction to boost the Air \glsentrytext{sphere} would mean a caster with Air 2 could cast a single spell as if they had Air 3.
  Using one in combat requires at least one \glsentrytext{ap} to grab it, and another to dispurse it into the air},
}

\longnewglossaryentry{ingredient}{
  name={Ingredients},
  text={Ingredient},
  prefix={an\space},
  nonumberlist,
  parent={alchemy},
  description={are the basic materials used to make any \glsentrytext{boon}, or \glsentrytext{talisman}, and for lots of medicines.
  Each has an elemental affinity, so a Fire Ingredient can only make a Fire \glsentrytext{boon}},
}

\longnewglossaryentry{elixir}{
  name={Elixirs},
  text={elixir},
  prefix={an\space},
  parent={alchemy},
  description={heal diseases.
  Each one requires a particular type of \glsentrytext{ingredient} to heal a particular disease},
}

\longnewglossaryentry{invocation}{
  name={Invocations},
  text={Invocation},
  prefix={an\space},
  parent={alchemy},
  description={are the basic sentence-formulae which define spells.
  They consist of one to five `descriptors', one action, and a target.
  For example, `{\sffamily Distant, Wax Air}' spells encourage air at a distance},
}

\longnewglossaryentry{talisman}{
  prefix={a\space},
  name={Talismans},
  text={Talisman},
  first={talisman (a one-use alchemical item)},
  firstplural={talismans (one-use alchemical items)},
  sort={Talisman},
  parent={alchemy},
  description={are spells, locked in an item, along with some activation condition.
  A talisman could open a magical gateway once it reaches a certain location, or bless the first person it sees with good luck.
  Many will strike the nearest, available target once activated, which makes them dangerous in the wrong hands},
}

\longnewglossaryentry{artefact}{
  name={Artefacts},
  text={Artefact},
  prefix={an\space},
  parent={alchemy},
  description={happen, often by accident, when someone imbues sentience into an unused \glsentrytext{talisman}, then leaves it to contemplate its existence for a century.
  spells given sentience, and function as long-term magical items.
  They frequently go awry, as they have a mind of their own, and their own wishes and values},
}

\longnewglossaryentry{cycle}{
  name={Cycles},
  text={cycle},
  prefix={a\space},
  description={As the \glsentrytext{ainumar} circles the Sun, \glsentrytext{fenestra} circles it.
  Each complete Cycle has twelve seasons, and which last three years in total},
}

\longnewglossaryentry{edge}{
  name={The Edge},
  sort={Edge},
  text={Edge},
  first={Edge of Civilization},
  prefix={an\space},
  category={location},
  parent={fenestra},
  description={lies one footstep off the \gls{lonelyRoad}, and surrounded every outer \glsentrytext{village}.
  Beyond this point, only dark forests, empty tundra, and hungry beasts wait.
  When people travel off-road, they have gone beyond the Edge},
}

\longnewglossaryentry{lonelyRoad}{
  name={The Lonely Road},
  sort={Lonely Road},
  text={lonely road},
  prefix={a\space},
  category={location},
  parent={fenestra},
  description={means any road between settlements.
  Going from one town to the next means a long journey through untamed territory},
}

\longnewglossaryentry{witch}{
  name={Witch},
  text={witch},
  plural={witches},
  prefix={a\space},
  parent={alchemy},
  description={simply means any spell-caster, often used as a derogatory term for a dishonoured \glsentrytext{doula}},
}

\longnewglossaryentry{witchcraft}{
  name={Witchcraft},
  text={Witchcraft},
  prefix={a\space},
  parent={action},
  description={is real magic, involving skill rather than relying on a recipe\ldots or any understanding of \glsentrytext{alchemy}},
}

\longnewglossaryentry{deep}{
  name={The Labyrinth},
  text={Labyrinth},
  sort={Labyrinth},
  prefix={a\space},
  category={location},
  parent={fenestra},
  description={is the network of frigid, nearly lifeless caverns, which sits beneath much of \glsentrytext{fenestra}},
}

\longnewglossaryentry{blight}{
  name={A Blight},
  text={blight},
  sort={Blight},
  prefix={a\space},
  category={location},
  parent={fenestra},
  description={is an area too full of goblins for anyone to live.
  Most blights have at least one cave or portal to the underground goblin realm},
}

\longnewglossaryentry{village}{
  name={Baileys},
  text={bailey},
  prefix={a\space},
  category={location},
  parent={fenestra},
  description={are walled villages, which stands beyond the protection of any towns, and endure attacks by wandering monsters.
  They mark the \glsentrytext{edge} of civilization, as nothing lies beyond them except the wild forest},
}

\longnewglossaryentry{bothy}{
  name={Bothies},
  text={bothy},
  plural={bothies},
  prefix={a\space},
  category={location},
  parent={lonelyRoad},
  description={are small half-way houses on long roads, built so that travellers can sleep safely after Sundown.
  Some have a single fireplace, and enough room for a half a dozen people and a donkey},
}

% Guilds

\longnewglossaryentry{sylf}{
  name={Sylf},
  category={god},
  symbol={\Aries},
  parent={ainumar},
  description={has griffin wings, with a woodspy's head, an arachnic thorax, and human belly; both chambers are painfully bloated from pregnancy.
  She gives birth to monsters endlessly.
  The \glsentrytext{guard} exist to either thwart, or feed her, dependng on whom one asks},
}

\longnewglossaryentry{templeOfBeasts}{
  name={The Temple of Beasts},
  text={Temple of Beasts},
  prefix={a\space},
  symbol={\glssymbol{sylf}},
  nonumberlist,
  description={The highest and lowest of all temples absorbs people with progressive political ideas, feckless drunks, dickheads, and scum.
  All of them become heroes, and forest-feed},
}

\longnewglossaryentry{broch}{
  name={Brochs},
  text={broch},
  prefix={a\space},
  category={location},
  parent={templeOfBeasts},
  description={ are grand towers which surround civilization.
  The \glsentrytext{guard} stay in them, playing loud pipes, and lighting fires to temp predators towards them.
  Once a predator arrives, the \gls{guard} can loose arrows towards them, meaning one less predator able to attack any nearby \glsentrytext{village}, or the inner settlements.
  A ring of flat earth surrounds each one, giving archers a clear shot at anything which emerges from the \glsentrytext{edge}},
}

\longnewglossaryentry{guard}{
  name={Night Guards},
  sort={Night Guard},
  text={Night Guard},
  prefix={a\space},
  parent={templeOfBeasts},
  nonumberlist,
  description={are the sorry lot who have nothing better to do than wander into the darkness and get eaten},
}

\longnewglossaryentry{jotter}{
  name={Jotters},
  text={jotter},
  prefix={a\space},
  parent={guard},
  description={do paperwork for the \glsentrytext{guard}, and control everything that their seniors don't care to manage},
}

\longnewglossaryentry{ranger}{
  name={Rangers},
  text={ranger},
  prefix={a\space},
  parent={guard},
  description={travel fast, often on horseback, to provide reinforcements to any \glsentrytext{village} in immediate trouble},
}

\longnewglossaryentry{thane}{
  name={Thanes},
  text={thane},
  prefix={a\space},
  parent={guard},
  description={have risen to the point where any \glsentrytext{jotter} will finally leave them alone.
  Most try to find some gold and retire at this point},
}

\longnewglossaryentry{abderian}{
  name={Abderian},
  category={god},
  symbol={$\Large\Circle$\hspace{-1em}\adfflowerleft},
  parent={ainumar},
  description={is the goddess of poison and rot.
  When she kills someone, she brings them to her banquet of pain, to see how long they can resist eating her rancid food},
}

\longnewglossaryentry{templeOfPoison}{
  name={The Temple of Poison},
  text={Temple of Poison},
  prefix={a\space},
  symbol={\glssymbol{abderian}},
  nonumberlist,
  description={Beer, brewing and baking protect humanity from \glsentrytext{abderian}, so this temple have established a divine monopoloy on the lot},
}

\longnewglossaryentry{wheatGuild}{
  name={The Wheat Guild},
  text={Wheat Guild},
  sort={Wheat Guild},
  prefix={a\space},
  category={location},
  parent={templeOfPoison},
  description={is where drinks are brewed, and drunks can drink.
  An image of a skeleton, tempting you to feast with her, reminds patrons why they should always eat at an official Wheat Guild hall},
}

\longnewglossaryentry{server}{
  name={Servers},
  text={Server},
  prefix={a\space},
  parent={templeOfPoison},
  description={make food for the \glsentrytext{wheatGuild} all day, and every \glsentrytext{warden} has one in their employ},
}

\longnewglossaryentry{eldren}{
  name={Eldren},
  category={god},
  symbol={\Sun},
  parent={ainumar},
  description={takes those who die of sickness or age.
  Many people save up their whole lives to be allowed into the \glsentrytext{healersGuild} once they become old, so they can die in peace, and go to his realm},
}

\longnewglossaryentry{templeOfSickness}{
  name={The Temple of Sickness},
  text={Temple of Sickness},
  prefix={a\space},
  symbol={\glssymbol{eldren}},
  nonumberlist,
  description={People only like one god.
  They all want \glsentrytext{eldren} to take them, rather than any alternatives, so anyone who has a little to spare puts it towards this temple, and hopes to see a bed inside one day, and die peacefully},
}

\longnewglossaryentry{healersGuild}{
  name={Healers' Guild},
  prefix={a\space},
  category={location},
  parent={templeOfSickness},
  description={take in the sick, disabled, or dying of \glsentrytext{fenestra}, where they tend to each other.
  Long-term staff always have some long-term disability, such as missing limbs.
  Many of the \glsentrytext{guard} start a new career here},
}

\longnewglossaryentry{helper}{
  name={Helpers},
  text={Helper},
  prefix={a\space},
  parent={templeOfSickness},
  description={tend to the sick and dying, on behalf of the \glsentrytext{healersGuild}.
  The temple only takes on new people from those with a disability of some kind, so the grounds are maximally accessible to everyone.
  The \glsentrytext{guard} who endure permanent injuries often retire as Helpers},
}

\longnewglossaryentry{mixer}{
  name={Mixers},
  text={mixer},
  prefix={a\space},
  parent={helper},
  description={can devise \glsentrytext{elixir} recipes to cure diseases.
  They create a constant demand on the \glsentrytext{ingredient}-trade, as people are always getting sick with one thing or another},
}

\longnewglossaryentry{counter}{
  name={Counters},
  text={counters},
  prefix={a\space},
  parent={helper},
  description={count the spare beds, pension-funds paid towards the \glsentrytext{templeOfSickness}, expected time until a sick person die, total \glsentrytext{ingredient}-count of each type, and expected total cures and corpses.
  They form the heart of every \glsentrytext{healersGuild}},
}

\longnewglossaryentry{nulla}{
  name={\qquad},
  category={god},
  prefix={a\space},
  parent={ainumar},
  symbol={$\Large\Circle$},
  description={\qquad},
}

\longnewglossaryentry{templeOfMisgenesis}{
  name={The Temple of Misgenesis},
  text={Temple of Misgenesis},
  prefix={a\space},
  symbol={$\Large\Circle$},
  nonumberlist,
  description={Some things never were.  This loose organization alters fortune to ensure nobody has to fail a task before they try, or give up on the best career they never considered},
}

\longnewglossaryentry{doula}{
  name={Doulas},
  text={Doula},
  prefix={a\space},
  parent={templeOfMisgenesis},
  description={help with births, blessings, and beginnings of all kind.
  They protect the population from misgeneis -- the death which occurs before life.
  Nobody begins a business venture or party without a blessing from a Doula},
}

\longnewglossaryentry{doulaShop}{
  name={Doula Shops},
  text={Doula Shop},
  sort={Doula Shop},
  prefix={a\space},
  category={location},
  parent={templeOfMisgenesis},
  symbol={$\Large\Circle$},
  description={sell various potions and blessings},
}

\longnewglossaryentry{paik}{
  name={Paik},
  category={god},
  symbol={$\Large\newmoon$},
  parent={ainumar},
  description={is the god of death by justice.
  When bandits swing from the noose in the \glsentrytext{court}, Paik takes them to his realm, and taunts them forever},
}

\longnewglossaryentry{templeOfJustice}{
  name={The Temple of Justice},
  text={Temple of Justice},
  prefix={a\space},
  symbol={\glssymbol{paik}},
  nonumberlist,
  description={Left to their own paranoia, people form mobs, and mob-justice prevails.
  This temple thwarts the worst plans of the god \glsentrytext{paik} by providing impartial, official, justice},
}

\longnewglossaryentry{court}{
  name={The Pit of Justice},
  text={Pit of Justice},
  prefix={a\space},
  category={location},
  sort={Pit of Justice},
  parent={templeOfJustice},
  description={is where a town's \glsentrytext{warden} resolves legal disputes, and decides on the correct punishment for criminals.
  All trials must be on display, to warn the people about the consequences of crime, so they can learn that justice always prevails},
}

\longnewglossaryentry{keeper}{
  name={Keepers},
  text={Keeper},
  nonumberlist,
  parent={templeOfJustice},
  prefix={a\space},
  description={bear the heavy burden of enforcing laws, and maintaining the \glsentrytext{court}.
  Every \glsentrytext{village} with a couple of hundred souls has a keeper to keep them right, and collect payments for the vital service they provide},
}

\longnewglossaryentry{warden}{
  name={Wardens},
  text={Warden},
  nonumberlist,
  parent={keeper},
  prefix={a\space},
  description={make the laws and decide fitting punishments for criminals in their \glsentrytext{court}},
}

\longnewglossaryentry{seneschal}{
  name={Seneschals},
  text={seneschal},
  parent={keeper},
  prefix={a\space},
  description={count every \glsentrytext{cp} in an estate, keep their \glsentrytext{warden} up-to-date on recent affairs, and keep in contact with every \glsentrytext{guard} \glsentrytext{jotter} in the area, and ensure the \glsentrytext{wheatGuild} don't cause problems},
}

\longnewglossaryentry{sunGuard}{
  name={The Sun Guard},
  sort={Sun Guard},
  text={Sun Guard},
  prefix={a\space},
  parent={keeper},
  description={are the upstanding soldiers who protect the city from all the unwashed masses, while wearing shiny-white tabards},
}

\longnewglossaryentry{sable}{
  name={Sable},
  category={god},
  symbol={$\large\hexstar$},
  parent={ainumar},
  description={releases cold into the world to watch people lay down and die in the snow, then takes them to his frigid realm, to place their frozen spirits there like an ornament},
}

\longnewglossaryentry{templeOfFrost}{
  name={The Temple of Frost},
  text={Temple of Frost},
  prefix={a\space},
  symbol={\glssymbol{sable}},
  nonumberlist,
  description={Cold saps strength slowly over \glsentrytext{Minquesta} and \glsentrytext{Alassea}, but this temple can push back the frost, denying \glsentrytext{sable} with clothing, a fire, and plenty of gossip},
}

\longnewglossaryentry{weaversGuild}{
  name={The Weavers' Guild},
  text={Weavers' Guild},
  sort={Weavers' Guild},
  prefix={a\space},
  category={location},
  parent={templeOfFrost},
  description={houses a massive fire, many looms, and plenty of space to tell a long story},
}

\longnewglossaryentry{weaver}{
  name={Weavers},
  text={Weaver},
  prefix={a\space},
  parent={templeOfFrost},
  description={produces the best and warmest of clothes for \glsentrytext{fenestra}'s harsh cold seasons},
}

\longnewglossaryentry{wrecan}{
  name={Wrecan},
  category={god},
  symbol={\Amor},
  parent={ainumar},
  description={is the goddess of hatred, bigotry, and vengeance.
  When people fight each other, she takes their soul to her realm of eternal war},
}

\longnewglossaryentry{templeOfHate}{
  name={The Temple of Hate},
  text={Temple of Hate},
  prefix={a\space},
  symbol={\glssymbol{wrecan}},
  nonumberlist,
  description={As resources run thin, humanity becomes its own worst enemy.
  Disgust spreads into bigotry, then war, as \glsentrytext{wrecan} pulls people into her realm},
}

\longnewglossaryentry{armourHall}{
  name={The Armourers' Guild},
  text={Armourers' Guild},
  prefix={an\space},
  category={location},
  sort={Armourer's Guild},
  parent={templeOfHate},
  description={take in angry, young people, and redirect their anger into work and negotiation skills.
  They handle all the major disputes which nobody wants to take to the \glsentrytext{templeOfJustice}, and often become the de-facto arbiters when one \glsentrytext{warden} cannot agree with another},
}

\longnewglossaryentry{armourer}{
  name={Armourers},
  text={Armourer},
  prefix={an\space},
  parent={templeOfHate},
  description={work in the \glsentrytext{armourHall} and guard civilization from death by Hate, with a combination of excellent armour, and diplomatic skills},
}

\longnewglossaryentry{yonder}{
  name={Yonder},
  category={god},
  symbol={$\Large\bell$},
  parent={ainumar},
  description={is the god who kills by curiosity.
  When idiots go to investigate something which sensible people would leave alone, they say their soul goes to live with \glsentrytext{yonder}.
  Nobody knows what happens after that, and there's only one way to find out\ldots},
}

\longnewglossaryentry{templeOfCuriosity}{
  name={The Temple of Curiosity},
  text={Temple of Curiosity},
  prefix={a\space},
  symbol={\glssymbol{yonder}},
  nonumberlist,
  description={People who go searching for answers often don't come back.
  This temple keeps an official log of all curiosities so that people don't have to go anywhere dangerous to learn -- they can just read.
  Anyone not content to stay inside and read eventually goes to write a travel-book},
}

\longnewglossaryentry{paperGuild}{
  name={The Paper Guild},
  text={Paper Guild},
  prefix={a\space},
  category={location},
  sort={Paper Guild},
  parent={templeOfCuriosity},
  description={makes maps, candles, paper, and soap.
  They always need more basilisk bodies to make the latter two},
}

\longnewglossaryentry{scribe}{
  name={Scribes},
  text={Scribe},
  prefix={a\space},
  parent={templeOfCuriosity},
  description={are anyone who works for the \glsentrytext{paperGuild}.
  Most spend their time making books or soap},
}

\longnewglossaryentry{seeker}{
  name={Seekers},
  text={Seeker},
  prefix={a\space},
  %seealso={paperGuild},
  parent={scribe},
  description={travel to deliver messages, and gather information for the \glsentrytext{paperGuild}},
}

\longnewglossaryentry{cartographer}{
  name={Cartographers},
  text={cartographer},
  prefix={a\space},
  %seealso={paperGuild},
  parent={scribe},
  description={create, confirm, and update maps.
    The most valuable of these contain information on \glsentrytext{ingredient}-rich ground, which they use for \glsentrytext{alchemy} spells to learn about the landscape, so they can make more maps},
}

%%%%% Seasons

\longnewglossaryentry{Qualmea}{
  name={Qualmea},
  prefix={a\space},
  nonumberlist,
  sort={1},
  parent={cycle},
  description={ -- \glsentrytext{fenestra}'s first season, when the ground quakes, and the \glsentrytext{ainumar} creates a long eclipse},
}

\longnewglossaryentry{Atya}{
  name={Atya},
  prefix={an\space},
  nonumberlist,
  sort={2},
  parent={cycle},
  description={ -- \glsentrytext{fenestra}'s second season},
}

\longnewglossaryentry{Alassea}{
  name={Alassea},
  prefix={an\space},
  nonumberlist,
  sort={3},
  parent={cycle},
  description={ -- \glsentrytext{fenestra}'s third season, when the wind bites.
  The Sun moves far away, and then the \gls{ainumar} blocks it, plunging the world into darkness.
  During this time, Water spells become more powerful, and becoming larger, or reshaping the water into elements from the caster's past},
}

\longnewglossaryentry{Cantea}{
  name={Cantea},
  prefix={a\space},
  nonumberlist,
  sort={4},
  parent={cycle},
  description={ -- \glsentrytext{fenestra}'s fourth season},
}

\longnewglossaryentry{Calea}{
  name={Calea},
  prefix={a\space},
  nonumberlist,
  sort={5},
  parent={cycle},
  description={ -- \glsentrytext{fenestra}'s fifth season, when the Sun shines hot, and basilisks wander even at night},
}

\longnewglossaryentry{Verea}{
  name={Verea},
  prefix={a\space},
  nonumberlist,
  sort={6},
  parent={cycle},
  description={ -- \glsentrytext{fenestra}'s sixth season},
}

\longnewglossaryentry{Otsea}{
  name={Otsea},
  prefix={an\space},
  nonumberlist,
  sort={7},
  parent={cycle},
  description={ -- \glsentrytext{fenestra}'s seventh season, when earthquakes destroy walls},
}

\longnewglossaryentry{Toldea}{
  name={Toldea},
  prefix={a\space},
  nonumberlist,
  sort={8},
  parent={cycle},
  description={ -- \glsentrytext{fenestra}'s eigth season},
}

\longnewglossaryentry{Laiquea}{
  name={Laiquea},
  prefix={a\space},
  nonumberlist,
  sort={9},
  parent={cycle},
  description={ -- \glsentrytext{fenestra}'s ninth season, when the Sun scorches the earth},
}

\longnewglossaryentry{Quainea}{
  name={Quainea},
  prefix={a\space},
  nonumberlist,
  sort={10},
  parent={cycle},
  description={ -- \glsentrytext{fenestra}'s tenth season},
}

\longnewglossaryentry{Minquesta}{
  name={Minquesta},
  prefix={a\space},
  nonumberlist,
  sort={11},
  parent={cycle},
  description={ -- \glsentrytext{fenestra}'s eleventh season, and second dark frost, as the \glsentrytext{ainumar} creates another eclipse},
}

\longnewglossaryentry{Ohta}{
  name={Ohta},
  prefix={an\space},
  nonumberlist,
  sort={12},
  parent={cycle},
  description={ -- \glsentrytext{fenestra}'s twelfth and final season, before the \glsentrytext{cycle} starts again},
}


%%%%% Symbols

\newglossaryentry{plant}{
  type=symbols,
  sort=TP,
  nonumberlist,
  name={\adfhangingflatleafleft},
  description={Plant, probably dangerous},
}

\newglossaryentry{encumbrance}{
  type=symbols,
  sort=AE,
  nonumberlist,
  name={\Semisextile},
  description={Encumbrance},
}

\newglossaryentry{N}{
  type=symbols,
  sort=RZ,
  nonumberlist,
  name={\Hygiea},
  description={Goblinoid},
}

\newglossaryentry{R}{
  type=symbols,
  sort=TM,
  nonumberlist,
  name={\NorthNode},
  description={Morphed creature},
}

\newglossaryentry{D}{
  type=symbols,
  sort=TU,
  nonumberlist,
  name={\Lilith},
  description={Undead creature},
}

\newglossaryentry{T}{
  type=symbols,
  sort=TT,
  nonumberlist,
  name={\Opposition},
  description={A team of multiple creatures},
}

\newglossaryentry{E}{
  type=symbols,
  sort=SN,
  nonumberlist,
  name={\Mercury},
  description={Sentient (any gender or none)},
}

\newglossaryentry{F}{
  type=symbols,
  sort=SF,
  nonumberlist,
  name={\Venus},
  description={Female},
}

\newglossaryentry{M}{
  type=symbols,
  sort=SM,
  nonumberlist,
  name={\Mars},
  description={Male},
}

\newglossaryentry{A}{
  type=symbols,
  sort=TA,
  nonumberlist,
  name={\Taurus},
  description={Animal},
}

% RACES

\newglossaryentry{Dw}{
  type=symbols,
  sort=RDwarf,
  nonumberlist,
  name={\Vulkanus},
  description={Dwarf},
}

\newglossaryentry{El}{
  type=symbols,
  sort=RElf,
  nonumberlist,
  name={\Moon},
  description={Elf},
}

\newglossaryentry{Hu}{
  type=symbols,
  sort=RHuman,
  nonumberlist,
  name={\Saturn},
  description={Human},
}

\newglossaryentry{Gn}{
  type=symbols,
  sort=RGnome,
  nonumberlist,
  name={\Kronos},
  description={Gnome},
}

\newglossaryentry{Nl}{
  type=symbols,
  sort=RGnoll,
  nonumberlist,
  name={\Admetos},
  description={Gnoll},
}

\newglossaryentry{morning}{
  type=symbols,
  sort=T1,
  nonumberlist,
  name={\LEFTcircle},
  description={is first light, first \glsentrytext{interval}},
}

\newglossaryentry{afternoon}{
  type=symbols,
  sort=T2,
  nonumberlist,
  name={\Circle},
  description={is the second \glsentrytext{interval}},
}

\newglossaryentry{evening}{
  type=symbols,
  sort=T3,
  nonumberlist,
  name={\RIGHTcircle},
  description={begins darkness, in the day's third \glsentrytext{interval}},
}

\newglossaryentry{night}{
  type=symbols,
  sort=T4,
  nonumberlist,
  name={\CIRCLE},
  description={means darkness over the fourth \glsentrytext{interval}},
}

\newglossaryentry{squash}{
  type=symbols,
  sort=ASquash,
  nonumberlist,
  name={\Vesta},
  description={Play this \gls{sq} at the same time as the available one (i.e. not from the same \gls{sq})},
}

\newglossaryentry{sqn}{
  type=symbols,
  sort=Asqn,
  nonumberlist,
  name={\Square},
  description={Side Quest is not ready yet},
}

\newglossaryentry{sqr}{
  type=symbols,
  sort=Asqr,
  nonumberlist,
  name={\CheckedBox},
  description={Side Quest is ready},
}

% SHORT COMMANDS

\newcommand{\T}[1][1]{\gls{T}\setcounter{noAppearing}{#1}} % groups
\newcommand{\M}{\gls{M}} % male
\newcommand{\F}{\gls{F}} % female
\newcommand{\E}{\gls{E}} % sentient
\newcommand{\A}{\gls{A}} % creature
\newcommand{\Pl}{\gls{plant}} % plant
\newcommand{\Sw}{\glssymbol{swarm}} % swarm
\newcommand{\N}{\gls{N}} % nura
\newcommand{\R}{\gls{R}} % morph
\newcommand{\D}{\gls{D}} % undead
\newcommand{\Dw}{\gls{Dw}} % dwarf
\newcommand{\El}{\gls{El}} % elf
\newcommand{\Hu}{\gls{Hu}} % human
\newcommand{\Gn}{\gls{Gn}} % gnome
\newcommand{\Nl}{\gls{Nl}} % gnoll
\newcommand{\squash}{\gls{squash}} % multi-part side quest
\newcommand{\sqr}{\glsentrytext{sqr}} % multi-part side quest
\newcommand{\sqn}{\glsentrytext{sqn}} % multi-part side quest

