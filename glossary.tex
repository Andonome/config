\makeglossaries

% We only want to show full descriptions (e.g. 'PC (Player Character)') in some books. These books make the file '.switch-gls', but most use nothing.

\IfFileExists{.switch-gls}{
  \setabbreviationstyle[acronym]{long-short-sc-desc}
}{
  \setabbreviationstyle[acronym]{nolong-short-sc}
}

\newacronym[
  description={The spell's current level (including Enhancements)},
  text={spell level},
  nonumberlist,
]{lv}{Lv}{Spell Level}

\newacronym[
  description={The smallest unit of currency},
  shortplural={cp},
  text={copper piece},
  longplural={copper pieces},
  prefix={a\space},
]{cp}{cp}{Copper Piece}

\newacronym[
  description={One silver piece is worth one hundred copper pieces},
  text={silver piece},
  shortplural={sp},
  longplural={silver pieces},
  prefix={a\space},
  ]{sp}{sp}{Silver Piece}

\newacronym[
  description={One gold piece is worth ten silver pieces, or one thousand copper pieces},
  shortplural={gp},
  longplural={gold pieces},
  text={gold piece},
  prefix={a\space},
]{gp}{gp}{Gold Piece}

\newglossaryentry{gm}{
  nonumberlist,
  prefix={a\space},
  name={The Judge},
  text={Judge},
  sort={Judge},
  description={runs the game, interprets the rules, and forgets to bring enough pencils}
}

\newglossaryentry{storypoint}{
  name={Story Points},
  text={Story Point},
  prefix={a\space},
  nonumberlist,
  description={allow players to declare that some part of their backstory arrives on scene to help the situation. They also grant \glsfmtplural{xp}}
}

\newglossaryentry{characterPool}{
  name={character pool},
  prefix={a\space},
  nonumberlist,
  description={the collection of total playable characters, including allies summoned with \glsentrytext{storypoint}}
}

\newglossaryentry{interval}{
  nonumberlist,
  prefix={an\space},
  name={An Interval},
  text={Interval},
  sort={Interval},
  description={means quarter of a day -- morning, afternoon, evening, or night}
}

\newglossaryentry{area}{
  name={Area},
  prefix={an\space},
  description={The basic unit of large spaces. An area is a space made distinct by its features. In a dungeon, each room might count as an area, while out in the open plains a forest might be composed of the local areas: `the centre with the big, felled tree; the river's fork; the priestess's house and the griffins' nesting site}
  }

\newglossaryentry{quickaction}{
  name={Response Action},
  prefix={a\space},
  first={\textit{Response Action}},
  description={An action taken as a reaction to being attacked, such as responding with an attack, or fleeing},
  }

\newglossaryentry{attribute}{
  name={Attribute},
  prefix={an\space},
  first={\textit{Attribute}},
  description={One of the six Traits which form the basis of the character -- Strength, Speed, et c}
  }

\newglossaryentry{skill}{
  name={Skill},
  prefix={a\space},
  nonumberlist,
  first={\textit{Skill}},
  description={Some training a character has, allowing them to be particularly good at one sort of profession or activity}
  }

\newglossaryentry{natural}{
  name={Natural Roll},
  prefix={a\space},
  nonumberlist,
  description={A natural roll is a roll where the physical dice land on some number. For example, a `natural 2' is where both dice come up facing 1, as opposed to a player gaining the result `2' from rolling a 3 and getting a -1 penalty. Similarly a `natural 12' is when the dice land on a `12' without modification}
  }

\newglossaryentry{round}{
  name={Round},
  prefix={a\space},
  nonumberlist,
  description={A round is an abstract measurement of time during which characters can make a series of attacks or cast spells. Each new round players adjust their combat tactics}
  }

\newglossaryentry{step}{
  name={Step},
  prefix={a\space},
  description={An abstract unit of measurement. We can imagine it about a yard long, or as wide as the step on your gaming board, or any other length. A more story-based game, without a board, might imagine each step is a `zone' or area in a room -- it matters little so long as each step is a consistent size}
  }

\newglossaryentry{trait}{
  name={Trait},
  prefix={a\space},
  description={Any gaming stat, such as a character's maximum MP, a Skill or an Attribute}
  }

\newglossaryentry{vitalShot}{
  name={Vitals Shot},
  prefix={a\space},
  description={An attack which strikes between the gaps in armour, dealing full Damage},
  }

\newglossaryentry{restingaction}{
  name={Resting Action},
  first={\textit{Resting Action}},
  prefix={a\space},
  description={An unhurried action, where a character can take time to do something right}
}

\newglossaryentry{downtime}{
  name={Downtime},
  prefix={a\space},
  description={These are the times between sessions, where characters train, study, or simply drink}
  }

\newacronym[
  description={A measure of how much luck the character has left, used solely to avoid Damage},
  shortplural={FP},
  prefix={an\space},
  ]{fp}{FP}{Fate Point}

\newacronym[
  description={The simple, linear, measure of a character's health and injury},
  shortplural={HP},
  prefix={a\space},
  ]{hp}{HP}{Health Point}

\newglossaryentry{fatigue}{
  name={Fatigue Points},
  text={Fatigue Point},
  nonumberlist,
  prefix={a\space},
  description={measure how tired a character is.
  Characters can put up with a number of Fatigue Points equal to their \glsentrytext{hp}, after which they receive penalties to act}
}

\newacronym[
  description={The number players need to roll on the dice to achieve a tie with the task. Rolling higher indicates they have their prize, rolling lower means some nasty outcome is upon them, and rolling a tie means both (or neither)},
  shortplural={TNs},
  prefix={a\space},
  ]{tn}{TN}{Tie Number}

\newacronym[
  description={The ``battery power'' of a magic user, which allows them to power spells},
  shortplural={MP},
  prefix={a\space},
  ]{mp}{MP}{Mana Point}

\newacronym[
  description={A measure of actions someone can take in a round, based on how fast they can move and react},
  shortplural={AP},
  prefix={an\space},
  ]{ap}{AP}{Action Point}

\newacronym[
  description={An abstract measurement of how much valuable experience and learning characters have acquired.  PCs spend XP to buy Traits},
  shortplural={XP},
  prefix={an\space},
  ]{xp}{XP}{Experience Point}

\newacronym[
  description={-- one of the characters run by the people playing the game},
  prefix={a\space},
  ]{pc}{PC}{Player Character}

\newacronym[
  description={magical shielding from the force sphere},
  shortplural={SP}
  prefix={a\space},
  ]{SP}{SP}{Shield Point}

\newacronym[
  description={A rating of protection, generally from wearing armour},
  shortplural={DR}
  prefix={a\space},
  ]{dr}{DR}{Damage Resistance}

\newacronym[
  description={Non Player Character -- anyone in the world played by the \glsentrytext{gm} rather than a player},
  prefix={a\space},
  ]{npc}{NPC}{Non Player Character}


\newglossaryentry{weight}{
  name={Weight Rating},
  text={Weight},
  nonumberlist,
  first={Weight Rating},
  prefix={a\space},
  description={measures how heavy something is when compared to a character's Strength Bonus.
  Creatures have a \glsentrytext{weight} equal to their own \glsentrytext{hp}}
}


\newglossaryentry{edge}{
  name={The Edge},
  sort={Edge},
  text={Edge},
  nonumberlist,
  first={Edge of Civilization},
  prefix={an\space},
  description={lies one footstep past the last walled town.
  Beyond this point, only dark forests, empty tundra, and hungry beasts wait.
  When people travel off-road, or have to push back the walls of the forest to prepare for civilization's expansions, they have gone beyond the Edge}
}

\newglossaryentry{deep}{
  name={The Deep},
  prefix={a\space},
  text={Deep},
  first={Deep Underground},
  prefix={a\space},
  description={is the area below or beyond the civilized caverns where the dwarves live, where most tunnels lie barren, and others contain nothing but dangerous beasts}
}

\newglossaryentry{guard}{
  name={Night Guard},
  text={Night Guard},
  prefix={a\space},
  nonumberlist,
  description={are the sorry lot who have nothing better to do than wander into the darkness and get eaten}
}

\newglossaryentry{blight}{
  name={A Blight},
  text={blight},
  sort={Blight},
  prefix={a\space},
  first={nura \textit{blight}},
  description={is is an area too full of nura for anyone to live.
  Most blights have at least one portal to the nura realm}
}

\newglossaryentry{ainumar}{
  name={Ainumar},
  plural={Ainumari},
  description={is a great orb in the sky, commonly supposed to be where the gods live}
}

\newglossaryentry{fenestra}{
  name={Fenestra},
  nonumberlist,
  description={is the world the characters exist}
}

%%%%% Symbols

\newglossaryentry{N}{
  type=symbols,
  sort=Nura,
  nonumberlist,
  name={\Hygiea},
  description={Nura (ogre, goblin, et c.)}
}

\newglossaryentry{D}{
  type=symbols,
  sort=Undead,
  nonumberlist,
  name={\Lilith},
  description={Undead creature}
}

\newglossaryentry{T}{
  type=symbols,
  sort=Team,
  nonumberlist,
  name={\Opposition},
  description={A team of multiple creatures}
}

\newglossaryentry{F}{
  type=symbols,
  sort=Female,
  nonumberlist,
  name={\Venus},
  description={Female}
}

\newglossaryentry{M}{
  type=symbols,
  sort=Male,
  nonumberlist,
  name={\Mars},
  description={Male}
}

\newglossaryentry{E}{
  type=symbols,
  sort=Sentient,
  nonumberlist,
  name={\Mercury},
  description={Sentient (any gender or none)}
}

\newglossaryentry{A}{
  type=symbols,
  sort=Animal,
  nonumberlist,
  name={\Taurus},
  description={Animal}
}

% RACES

\newglossaryentry{Dw}{
  type=symbols,
  sort=Dwarf,
  nonumberlist,
  name={\Vulkanus},
  description={Dwarf}
}

\newglossaryentry{El}{
  type=symbols,
  sort=Elf,
  nonumberlist,
  name={\Moon},
  description={Elf}
}

\newglossaryentry{Hu}{
  type=symbols,
  sort=Human,
  nonumberlist,
  name={\Saturn},
  description={Human}
}

\newglossaryentry{Gn}{
  type=symbols,
  sort=Gnome,
  nonumberlist,
  name={\Kronos},
  description={Gnome}
}

\newglossaryentry{Nl}{
  type=symbols,
  sort=Gnoll,
  nonumberlist,
  name={\Admetos},
  description={Gnoll}
}

\newglossaryentry{squash}{
  type=symbols,
  sort=Squash,
  nonumberlist,
  name={\Vesta},
  description={Play Side Quest at the same time as the next}
}

\newglossaryentry{sqn}{
  type=symbols,
  sort=sqn,
  nonumberlist,
  name={\Square},
  description={Side Quest is not ready yet}
}

\newglossaryentry{sqr}{
  type=symbols,
  sort=sqr,
  nonumberlist,
  name={\CheckedBox},
  description={Side Quest is ready}
}

% SHORT COMMANDS

\newcommand{\T}[1][1]{\gls{T}\setcounter{noAppearing}{#1}} % groups
\newcommand{\M}{\gls{M}} % male
\newcommand{\F}{\gls{F}} % female
\newcommand{\E}{\gls{E}} % sentient
\newcommand{\A}{\gls{A}} % creature
\newcommand{\N}{\gls{N}} % nura
\newcommand{\D}{\gls{D}} % undead
\newcommand{\Dw}{\gls{Dw}} % dwarf
\newcommand{\El}{\gls{El}} % elf
\newcommand{\Hu}{\gls{Hu}} % human
\newcommand{\Gn}{\gls{Gn}} % gnome
\newcommand{\Nl}{\gls{Nl}} % gnoll
\newcommand{\squash}{\gls{squash}} % multi-part side quest
\newcommand{\sqr}{\gls{sqr}} % multi-part side quest
\newcommand{\sqn}{\gls{sqn}} % multi-part side quest
