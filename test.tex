\documentclass[a4paper,openany]{book}
\usepackage{bind}
\usepackage{lipsum}

\date{\today}
\makeindex[name=spells,title={Spell Summaries},columns=2]

\settoggle{debug}{true}

\begin{document}

\frontmatter

\chapter{Random Stuff}

\section{Introduction}

\begin{multicols}{2}

\subsection{This Document}

This is a test document, to make sure new code works before sticking it in a project.
Current day should equal \arabic{r4b}.
That means the season in Fenestra is \showSeason, which means it's \showTemperature.

\npc{\M}{Random Guy}
\person{1}% STRENGTH
{1}% DEXTERITY 
{1}% SPEED
{{-2}% INTELLIGENCE
{-1}% WITS
{0}}% CHARISMA
{0}% DR
{1}% COMBAT
{Academics 1, Wyldcrafting 1
}% SKILLS
{\Dagger, pieces of string}% EQUIPMENT
{
  \setcounter{Fate}{2}
}

\ifodd\value{diceNo}

\spell{Bubble}% Name
  {}% Enhancements
  {Warp}% Action
  {Air}% Spheres
  {the density of the liquid}% Resist with
  {A small air-bubble surrounds the caster, protecting them from foul gasses and allowing them to go underwater}% Description
  {While mostly invisible, the bubble is real, and will go nowhere, resisting all but the strongest winds.

    Casters often use this to protect themselves from Noxious Fog spells.}

\spell{Noxious Fog}% Name
  {}% Enhancements
  {Wane}% Action
  {Air}% Spheres
  {the wind speed}% Resist with
  {A pocket of foul air inflicts \arabic{spellPlusTwo}~\glspl{fatigue} each round, to anyone who inhales it}% Description
  {

    The fog only covers \pgls{step} -- enough for a doorway, or small tunnel.
    It dissipates after a couple of turns, or less if wind is present.

    Casting the spell outdoors raises the \gls{tn} by 1, casting it on a windy day raises it by 3, and casting the spell in a storm raises the \gls{tn} by 5.
  }

\else

\spell{Wind Chill}% Name
  {}% Enhancements
  {Wax}% Action
  {Air}% Spheres
  {\roll{Strength}{Athletics}}% Resist with
  {As the caster shouts `\textit{back}', a blast of air pushes the target back, and they lose \arabic{spellPlusTwo}~\glspl{ap} minus their Strength Bonus}% Description
  {}

\fi
\stepcounter{diceNo}


\begin{speechtext}

  ``Would you tell me, please, which way I ought to go from here?''

  ``That depends a good deal on where you want to get to.''

\end{speechtext}

\subsection{And now for something completely different}

\Glspl{guard} and \glspl{bothy} and \glspl{village}, oh my!

Also, \gls{talisman}, \gls{paperGuild}, \gls{weaver}, \gls{court}, and \glspl{storypoint}.

\begin{boxtable}

Roll & Result \\\hline

12 & Success \\

11 & Failure \\

\end{boxtable}

\subsection{Random Text}

\lipsum[7]

\subsection{And further more\ldots}

\lipsum[10]

\begin{boxtext}
  \lipsum[4]
\end{boxtext}

\end{multicols}

\mainmatter

\chapter{Side Quests}

\declareSQareas{Slums,Artgate}

\renewcommand\sqarea{Slums}

Featuring losing \glspl{hp} beyond the \gls{edge}.

\begin{multicols}{2}

\sidequest{Cleaning Up}

\lipsum[2]

\sqpart{Slums}% AREA
{Noisy Neighbours}% NAME
{who celebrate their new baby all week}% SUMMARY

\lipsum[3]

\sqpart{Slums}% AREA
{The Rally}% NAME
{for equal rights for squirrels}% SUMMARY

\lipsum[3]

\sqpart{Artgate}% AREA
{The Proposal}% NAME
{Countess Bleachclaw wants a quieter town}% SUMMARY

\lipsum[3]

\stopcontents[sq]

\end{multicols}

\printglossary

\printAllSideQuests{Slums,Artgate}

\chapter{Stat Blocks}

\section{Humanoids}

\begin{multicols}{2}

\subsection{Humans}

\humanfarmer

\humanmaid

\humansoldier

\humanarcher

\royalguard

\story{4}{this One Time, at Bandcamp}

\humansoldier

\humandiplomat[\NPC{\Hu\F}{Bob}{Erratic}{Snaps fingers}{Paladin}]

This section is non-verbose, so only basic stats are shown.
\settoggle{verbose}{false}
\settoggle{debug}{false}

\basilisk

\humanbard[\renewcommand\npcQuote{We are all going to die}\NPC{\Hu\M}{Chris}{Despondent}{Sighs}{Jester}]

\humanbard

\humanthief

\humanalchemist

\necromancer

\subsection{Dwarves}

\settoggle{verbose}{true}

\dwarvensoldier

\dwarventrader

\dwarvenrunemaster

\subsection{Elves}

\elf

\elf

\elvenenchanter

\dryad

\subsection{Gnomes}

\gnome

\gnomishsoldier

\gnomishsoldier

\gnomishillusionist

\subsection{Gnolls}

\gnollhunter

\gnollshaman

\showStdSpells

\gnollshaman

\showStdSpells[\setcounter{enc}{1}]

\showStdSpells[
  \spell{Permanent Bubbles}% Name
    {}% Enhancements
    {Warp}% Action
    {Air}% Spheres
    {wind speed}% Resist with
    {The caster gives a pep-talk to a little bubble of water, and it remains as solid as paper, floating into the air and growing to the size of a dog}% Description
    {All bubbles die in the end, but these bubbles have been spotted for miles away from their caster before dying}
]

\end{multicols}

\section{Forest Critters}

\begin{multicols}{2}

\bear

\boar

\chitincrawler

\basilisk

\wolf

\woodspy

\end{multicols}

\section{Undead}

\begin{multicols}{2}

\ghoul

\ghast

\demilich

\lich

\end{multicols}

\chapter{Nura}

\begin{multicols}{2}

\subsection{Humanoids}

\goblin[\npc{\N}{Random Goblin}]

\goblin

\goblin

\goblincaster

\hobgoblin

\ogre

\deepogre

\subsection{Morph Animals}

\morphrat

\morphhorse

\morphcrab

\morphcat

\morphslug

\morphspider

\morphwolf

\end{multicols}

\chapter{Outsiders}

\section{Weird Ones}

\begin{multicols}{2}

\archmage

\archmage

\boxPair[t]{
  \subsubsection{Box Pairs}
  This is the \texttt{\textbackslash boxPair} command.
  It lets a pair of anything sit at the bottom of the page.
  The first argument sits in the left column, and the next in the right.

  The command works well if you have a creature box, like this lava-man.
}{
    \lavaman
}

\dragon

\rockman

\lavaman

\end{multicols}

\section{Same}

\begin{multicols}{2}

\rockman

\archmage

\dragon

\archmage

\lavaman

\end{multicols}

\section{Talismans}

\begin{multicols}{2}

\scrollOfFriendship

\bloodCandle

\minersWeed

\end{multicols}

%%%%% Test Character Sheet

\setcounter{Strength}{0}
\setcounter{Dexterity}{-1}
\setcounter{Speed}{0}
\setcounter{Intelligence}{1}
\setcounter{Wits}{0}
\setcounter{Charisma}{1}

\renewcommand\concept{Loner}
\renewcommand\race{Human}
\renewcommand\rank{Novice}
\renewcommand\name{Proskuff}
\renewcommand\code{Conqueror}

\setcounter{Academics}{2}
\setcounter{Athletics}{0}
\setcounter{Caving}{0}
\setcounter{Crafts}{0}
\setcounter{Deceit}{0}
\setcounter{Empathy}{0}
\setcounter{Medicine}{0}
\setcounter{Performance}{0}
\setcounter{Larceny}{0}
\setcounter{Seafaring}{0}
\setcounter{Stealth}{0}
\setcounter{Tactics}{1}
\setcounter{Vigilance}{0}
\setcounter{Wyldcrafting}{0}

\setcounter{Air}{2}
\setcounter{Combat}{0}
\setcounter{Projectiles}{0}

\setcounter{fp}{5}

\renewcommand\characterWeapon{\shortsword}
\renewcommand\characterArmour{}
\renewcommand\characterEquipment{Bag of flour, bag of chalk, dagger, unopened letter}
\setcounter{gold}{0}

\settoggle{examplecharacter}{true}
\thispagestyle{empty}
\begin{tcbposter}[
  coverage = {
      spread,
  },
  poster   = {
    showframe=false,
    columns=30,
    rows=18
  },
  boxes    = {
    enhanced standard jigsaw,
    boxsep=2pt,
    left=1pt,
    right=1pt,
    boxrule=.6mm,
    colback=white,
    drop fuzzy shadow,
   }
]

% reset damage so it calculates properly
\setcounter{damage}{0}
\setcounter{weaponBonus}{0}
\setcounter{knacks}{0}

%----
  \posterbox[
    remember, blankest, halign=center,valign=center,
  ]{name=title,column=1,span=29}{
    \vspace{.8cm}
    \begin{tabularx}{\linewidth}{lXlXlX}
      \hiderowcolors
      \textbf{Name:} & \iftoggle{examplecharacter}{\sffamily\name}{\dotfill} &
      \textbf{Player:} & \dotfill &
      \textbf{Code:} & \iftoggle{examplecharacter}{\sffamily\code}{\dotfill}
      \\
      \\

      \textbf{Race:} & \iftoggle{examplecharacter}{\sffamily\race}{\dotfill} &
      \textbf{Concept:} & \iftoggle{examplecharacter}{\sffamily\concept}{\dotfill} &
      \textbf{Rank:} & \iftoggle{examplecharacter}{\sffamily\rank}{\dotfill} \\
    \end{tabularx}
}

%----
  \posterbox[
    adjusted title=Attributes \hint{ 5 | 10 / 20 / 30 / 50 },
    remember,
  ]{name=attributes,column=1,row=3,span=10,rowspan=4}{
    \hspace{-0.5em}
    \renewcommand{\arraystretch}{1.5}
    \addtolength{\tabcolsep}{-0.53em}
    \begin{tabularx}{\linewidth}{Ycccccccccc}
    & \tiny{-4} & \tiny{-3} & \tiny{-2} & \tiny{-1} & \tiny{0} & \tiny{1} & \tiny{2} & \tiny{3} & \tiny{4} \\[-5pt]
    \showAttribute{Strength}
    \showAttribute{Dexterity}
    \showAttribute{Speed}
    \showAttribute{Intelligence}
    \showAttribute{Wits}
    \showAttribute{Charisma}[-8pt]
    \hspace{2em}\footnotesize{\dicef{7}} & & \tiny{2} & \tiny{3} & \tiny{4--5} & \tiny{6--8} & \tiny{9--10} & \tiny{11} & \tiny{12} & \\
    \end{tabularx}
  }

%----

  \posterbox[
    remember, blankest,interior engine=path,valign=center,
  ]{name=gumption,column=1,row=9,span=4,rowspan=8}{

    \begin{tikzpicture}[
circle label/.style = {
        postaction={
            decoration={
                text along path,
                text = {#1},
                text align=center,
                reverse path,
            },
        decorate,
    }
    }
    ]
     \path[circle label={FP = XP / 10 + Charisma}] (-1,3) arc (-90:360-90:1.1);
      \draw[dotted] (-1,4) circle (1) node [align=left,text width=3.5em] {};
     \path[circle label={HP = 6 + Strength}] (-1,0.6) arc (-90:360-90:1.1);
      \draw[dotted] (-1,1.7) circle (1);
     \path[circle label={Fatigue Points}] (-1,-2) arc (-90:360-90:1.1);
      \draw[dotted] (-1,-0.9) circle (1);
     \path[circle label={MP = 3 x spheres}] (-1,-4.6) arc (-90:315-45:1.1);
      \draw[dotted] (-1,-3.5) circle (1) node [align=left] {};
     \path[circle label={Encumbrance}] (-1,-7.2) arc (-90:315-45:1.1);
      \draw[dotted] (-1,-6.1) circle (1) node [align=left] {};
    \end{tikzpicture}

  }
%----
  \posterbox[adjusted title={M\^{e}l\'ee \hint{ 10 / 20 / 40 }},
  remember,
  ]{name=melee,column=11,row=3,span=8,rowspan=4}{
    \renewcommand{\arraystretch}{1.4}
    \begin{tabularx}{\linewidth}{Xr@{}}
      \hiderowcolors
      \ifnum\value{Brawl}>0
        \skill{Brawl}
      \else
        \emptySkill
      \fi
      \ifnum\value{Combat}>0
        \skill{Combat}
      \else
        \emptySkill
      \fi
      \ifnum\value{Projectiles}>0
        \skill{Projectiles}
      \else
        \emptySkill
      \fi
      \ifnum\value{Air}>0
        \skill{Air}
      \else
        \emptySkill
      \fi
      \ifnum\value{Fire}>0
        \skill{Fire}
      \else
        \emptySkill
      \fi
      \ifnum\value{Earth}>0
        \skill{Earth}
      \else
        \emptySkill
      \fi
      \ifnum\value{Water}>0
        \skill{Water}
      \else
        \emptySkill
        \emptySkill
      \fi
    \end{tabularx}
  }

%----

  \posterbox[
  adjusted title=Items to Hand,
  remember,
  ]{name=armoury,
  column=5,
  row=9,
  span=22,
  rowspan=3}{
    \renewcommand{\arraystretch}{1.5}
    \vspace{-1em}
    \begin{tabularx}{\linewidth}{p{.37\textwidth}YYYY}
      \hiderowcolors
      \setlength{\parskip}{3mm}
      \textbf{Weapon} & \textbf{Bonus} & \textbf{Damage} & \textbf{\Glsentrytext{ap} Cost} & \textbf{Weight} \\
      
    \iftoggle{examplecharacter}{
      \hline
        \ifdefempty{\characterWeapon}{
          \\
          \\
        }{
          \sffamily\characterWeapon & \sffamily\arabic{weaponBonus} & \sffamily\arabic{damage} & \sffamily\arabic{heft} & \sffamily\arabic{weight} \\
        }
    }{
      \Repeat{2}{
        \hspace{3em} &
        \Repeat{3}{\statCircle} &
        \Repeat{3}{\statCircle} &
        \Repeat{3}{\statCircle} &
        \Repeat{5}{\Large\Square} \\
      }
    }
    \end{tabularx}

    \vspace{1em}
    \begin{tabularx}{\linewidth}{p{.4\textwidth}YYY}
      \hiderowcolors
      \hline
      \textbf{Armour} & \textbf{\Glsentrytext{dr}} & \textbf{Covering} & \textbf{Weight} \\
      \iftoggle{examplecharacter}{
        \ifdefempty{\characterArmour}{}{
          \sffamily\characterArmour & \sffamily\arabic{armourDR} & \sffamily\arabic{covering} & \sffamily\arabic{weight} \\
        }
      }{
        \hspace{3em} &
        \Repeat{5}{\statCircle} &
        \Repeat{5}{\statCircle} &
        \Repeat{5}{\Large\Square} \\
      }
    \end{tabularx}
  }

%-----
  \posterbox[
    blankest,interior engine=path, halign=center,valign=center,
    opacityback=0,
    remember,
  ]
  {name=derived,column=2,row=6,span=18,rowspan=4}{
    \addtolength{\tabcolsep}{-0.2em}
    \begin{tabularx}{\linewidth}{XXXX}
      \hiderowcolors
      \textbf{\Glsentrytext{dr} / Covering} & \textbf{\Glsentrytext{ap}} &\textbf{Attack} & \textbf{Damage} \\
      \vspace{3em}
      \\
      \\
      \\
      \iftoggle{examplecharacter}{\sffamily\arabic{dr}}{\underline{\hspace{2em}/\hspace{2em}}}
      &
      \iftoggle{examplecharacter}%
      {%
        \addtocounter{Speed}{3}%
        \sffamily\arabic{Speed}%
      }%
      {\underline{\hspace{2em}}}
      &
      \iftoggle{examplecharacter}{\sffamily}{}2D6%
      \iftoggle{examplecharacter}% Bonus
      {%
      \addtocounter{Combat}{\value{Dexterity}}%
      \addtocounter{Combat}{\value{weaponBonus}}%
        \sffamily\absNum{Combat}%
      }%
      {\underline{\hspace{2em}}} &
      \iftoggle{examplecharacter}% Damage
      {%
        \addtocounter{damage}{\value{Strength}}%
        \addtocounter{damage}{4}%
        \sffamily\dmg{damage}%
      }%
      {
        \underline{\hspace{1em}} D6+\underline{\hspace{1.5em}}
      }
      \\
      \tiny (by Armour) & \tiny 3 + Speed & \tiny 2D6 + Dex \par + Combat \par + Weapon Bonus & \tiny 1D6 + Strength \par + Weapon  \\
    \end{tabularx}
  }

%----

    \setcounter{track}{7}
    \posterbox[
    blankest,
    remember,
    ]{name=track,column=27,row=3,span=2.8,rowspan=15.8}{ 
      {\large
        \vspace{1em}
        \Repeat{14}{\tracker}
      }
      }


%-----
  \posterbox[
  adjusted title={Skills \hint{ 5 / 10 / 15 }},
  remember,
  ]
  {name=skills,column=19,row=3,span=8,rowspan=6}{
    \renewcommand{\arraystretch}{1.2}
    \begin{tabularx}{\linewidth}{Xr@{}}

      \hiderowcolors
      \skill{Academics}
      \skill{Athletics}
      \skill{Caving}
      \skill{Crafts}
      \skill{Deceit}
      \skill{Empathy}
      \skill{Medicine}
      \skill{Performance}
      \skill{Larceny}
      \skill{Seafaring}
      \skill{Stealth}
      \skill{Tactics}
      \skill{Vigilance}
      \skill{Wyldcrafting}
      \emptySkill
      \emptySkill

    \end{tabularx}
  }


%----
  \posterbox[
    adjusted title=Backpack \hint{\glsentrytext{tn} to remove: \weeline},
    remember,
  ]
  {name=equipment,
    column=6,
    row=15,
    span=23,
    rowspan=3.7,
  }{

    \iftoggle{examplecharacter}{\sffamily\characterEquipment\vspace{12em}}{
      \vspace{1.3em}
      \Repeat{5}{
        {\Repeat{5}{\Square}\hbox to .95\textwidth{\leaders\hbox to 30pt{\hss . \hss}\hfil}\vspace{1em}}
      }
    }
  }

%---

  \posterbox[
    adjusted title=Abilities \& Conditions,
    remember,
  ]
  {name=abilities,
  column=5,
  row=12,
  span=11,
  rowspan=3}{
    \iftoggle{examplecharacter}{}{
      \lineDots[4]
    }
  }

  \posterbox[
    adjusted title=Knacks \hint{ 5 / 10 / 15/ 20 },
    remember,
  ]
  {name=knacks,
  column=16,
  row=12,
  span=11,
  rowspan=3}{

  \setlength{\parskip}{0mm}

    \iftoggle{examplecharacter}{
      \sffamily\knackOne

      \sffamily\knackTwo
    }{
      \lineDots[4]
    }
  }

%------
  \posterbox[
    adjusted title=XP,
    remember,
  ]
  {name=xp,
  column=1,
  row=17,
  span=5,
  rowspan=1.7}{

  \setlength{\parskip}{0mm}
    \small Total:
    \iftoggle{examplecharacter}{\textcolor{gray}{50}}{}
    \tcblower
    \small Unspent:
  }


  \posterbox[
    remember, blankest, halign=center,valign=center,
  ]{name=money,column=6,span=10,row=18,rowspan=1}{
    \vspace{-1.1em}
    \begin{tabularx}{\hsize}{llX}
      \glsentryshortpl{cp}\weeline &
      \glsentryshortpl{sp}\weeline &
      \glsentryshortpl{gp}\weeline \\
    \end{tabularx}
  }

%------
\csComments

\end{tcbposter}

\settoggle{examplecharacter}{false}
\renewcommand\csComments{}
\clearpage


\chapter{Bestiary Chapters}

\begin{multicols}{2}

\settoggle{genExamples}{true}

When using a bestiary chapter, the stats appear as dice rolls, rather than fixed amounts.

\subsection{Humans}

\humanfarmer

\humansoldier

\royalguard

\humandiplomat

\humanbard

\humanthief

\humanalchemist

\humanalchemist

\necromancer

\subsection{Dwarves}

\dwarvensoldier

\dwarventrader

\dwarvenrunemaster

\subsection{Elves}

\elf

\elf

\elvenenchanter

\hag

\subsection{Gnomes}

\gnome

\gnomishillusionist

\subsection{Gnolls}

\gnollhunter

\gnollshaman

\gnollshaman

\end{multicols}

\section{Forest Critters}

\begin{multicols}{2}

\bear

\boar

\huntingdog

\cat

\chitincrawler

\basilisk

\end{multicols}

\section{Underground}

\begin{multicols}{2}

\umberhulk

\watcher

\jelly

\jelly

\jelly

\jelly

\end{multicols}

\section{Undead}

\begin{multicols}{2}

\ghoul

\ghast

\demilich

\lich

\end{multicols}

\section{Nura}

\begin{multicols}{2}

\subsection{Animals}

\morphhorse

\morphcrab

\morphcat

\morphslug

\morphspider

\morphwolf

\subsection{Humanoids}

\goblin

\goblincaster

\hobgoblin

\ogre

\end{multicols}

\settoggle{genExamples}{false}

\chapter{Lots of Text}

\begin{multicols}{2}

\noindent
\lipsum

\end{multicols}

\newcommand{\tests}{

\chapter{Another Test}

\section{Test Section}

\begin{multicols}{2}

\randomdozen
\setcounter{track}{\value{r12}}

\swarm[\flight]{Stirges}{\value{r12}}{1}{3}{1}

\lipsum[\arabic{r4}]

\humanbard

\lipsum[2]

\boxPair[t]{
  \subsubsection{Box Pairs}
  This is the \texttt{\textbackslash boxPair} command.
  It lets a pair of anything sit at the bottom of the page.
  The first argument sits in the left column, and the next in the right.

  The command works well if you have a creature box, like this griffin.
}{
    \griffin
}

\subsection{Test subsection}

\dragon

\settoggle{allyCharacter}{true}

\elf

\settoggle{examplecharacter}{true}

\lipsum[3]

\elf

\settoggle{allyCharacter}{false}

\subsubsection{Test sub-sub-section}

\lipsum[10]

\subsubsection{Test sub-sub-section again}

\subsection{Test subsection}

\lipsum[2]

\elf

\settoggle{examplecharacter}{false}

\lipsum[10]

\section{Last Section}

\lipsum[3]

\end{multicols}

}

\tableofcontents

\Repeat{2}{\tests}

\settoggle{verbose}{false}

\Repeat{2}{\tests}

\end{document}
